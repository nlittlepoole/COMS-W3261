\documentclass[]{article}
\usepackage{amsmath}
\usepackage{amssymb}
\usepackage{amsthm}
\begin{document}
\newtheorem{keyword}{Definition}
\newtheorem{example}{Example}

\title{COMS W3261 \\ Computer Science Theory \\ Lecture 1}
\author{Alexander Roth}
\date{2014 -- 09 -- 03}
\maketitle
\section{Schedule}
  \begin{itemize}
    \item Lectures: Mondays and Wednesdays, 1:10 -- 2:25pm, 833 Mudd.
    \item Office Hours: Mondays and Wednesdays, after class.
  \end{itemize}

\section{Course Objectives}
  \begin{itemize}
    \item Learning computational thinking
    \item Understanding the fundamental models of computation that underlie 
    modern computer hardware, software, and programming languages.
    \item Discovering that there are problems no computer can solve.
    \item Discovering that there are limits on how fast a computer can solve a 
    problem.
    \item Mastering the foundations of automata theory, computability theory, 
    and complexity theory.
    \item Learning about applications of computer science theory to algorithms, 
    programming languages, compilers, natural language translation, operating 
    systems, and software verification.
  \end{itemize}

\section{Course Syllabus}
  \begin{itemize}
    \item Languages and decision problems
    \item Finite automata
    \item Regular expressions
    \item Properties of regular languages
    \item Context-free grammars
    \item Pushdown automata
    \item Algorithms and Turing machines
    \item Lambda calculus
      \begin{itemize}
        \item The computational model for functional languages. Java 8 now has 
        lambdas.
      \end{itemize}
    \item Undecidability
    \item Complexity theory
  \end{itemize}

\section{Course Requirements}
  \begin{itemize}
    \item Homeworks (best four out of five homeworks will constitute 20\% of 
    final grade)
    \item Midterm (40\% of final grade)
    \item Final (40\% of final grade)
  \end{itemize}

\section{Languages}
  \begin{itemize}
    \item An \emph{alphabet} $\sum$ is a finite, nonempty set of symbols.
    \item A \emph{function} maps an element to a range.
    \item[\textbf{Example:}]
    $\{0, 1\}$, ASCII, Unicode. 
    \item A \emph{string} is a finite sequence of symbols chosen from some 
    alphabet.
    \item[\textbf{Example:}] $\varepsilon$, the empty string, 0, 01, 011
    \begin{itemize}
      \item Don't forget about the empty string, it's important!
    \end{itemize}
  \end{itemize}

  \subsection{Terms associated with strings}
    \begin{keyword}
    Prefix: Any sequence of characters at the beginning of a string.
    \end{keyword}
    \begin{example}
    ``dog'' has 4 prefixes: $\varepsilon$, d, do, dog.
    \end{example}
  All strings have $n + 1$ prefixes.

    \begin{keyword}
    Suffixes: Any sequence of characters at the end of a string.
    \end{keyword}
    \begin{example}
    ``dog'' has 4 suffixes. $\varepsilon$, g, go, god.
    \end{example}
  All strings have $n + 1$ suffixes

    \begin{keyword}
    Substrings: Obtained by maintaining the sequence of characters while 
    reducing the size of the strings.
    \end{keyword}
    \begin{example}
    ``dog'': \\
    Shortest String: $\varepsilon$ \\
    String of length 1: d, o, g \\
    String of length 2: do, og, dg \\
    String of length 3: dog
    \end{example}
  All substrings are of length $n! + 1$

    \begin{keyword}
    Subsequences: Obtained by deleting zero or more characters in a string
    \end{keyword}
    \begin{example}
    ``dog'': $\varepsilon$, d, o, g, do, dg, og, dog
    \end{example}
  $2^n$ subsequences in any given string $x$.\\
  A string is ordered, strings are sequences and sequences have orders.\\

  \textbf{Operations on Strings}
  Common operations:
  \begin{itemize}
    \item Concatenation
    \item Reversal
  \end{itemize}

  \textbf{Terms associated with Strings}
  \begin{itemize}
    \item Prefix
    \item Suffix
    \item Substring
    \item Subsequence
  \end{itemize}

  A \emph{language} over $\sum$ is a set of strings whose symbols are chosen 
  from $\sum$
  \begin{example}
    \begin{itemize}
      \item The empty set, $\varnothing$
      \item $\{ 0, 1 \}$, $\{0\}$, $\{1\}$, $\{01\}$
      \item $P = \{ 10, 11, 101, 111, 1011, 1101, \ldots \}$ (the binary 
      representation of the prime numbers)
    \end{itemize}
  \end{example}

  \textbf{Distinction between finite and infinite languages:}
    \begin{itemize}
      \item Distinction between finite and infinite languages:
      \begin{itemize}
        \item There are a finite number of objects in the set.
        \item Two types of infinite:
        \begin{itemize}
          \item \textbf{Countably:} 1 to 1 mapping between the numbers in the 
          set against the natural integers. A set of all syntactically valid 
          java programs is countably infinite.
          \item \textbf{Uncountably:} Do not have 1 to 1 mapping between the 
          natural integers and the given set.
        \end{itemize}
      \end{itemize}
    \end{itemize}

  \subsection{Kleene Closure}
    \[ L \cup M \] 
    \[ L \cap M \]
    \[ LM = \{ \, xy \, | x \in \]


\end{document}