\documentclass[]{article}
\usepackage{amsmath}
\usepackage{amssymb}
\usepackage{amsthm}
\usepackage{listings}
\usepackage{multirow}
\usepackage{tikz}
\usepackage{tikz-qtree}
\usepackage{tipa}
\usetikzlibrary{arrows,automata}
\begin{document}
\newcommand*{\xml}[1]{\texttt{<#1>}}

\title{COMS W3261 \\ Computer Science Theory \\ Lecture 8\\ Pushdown Automata}
\author{Alexander Roth}
\date{2014 -- 09 -- 29}
\maketitle

\section*{Outline}
  \begin{enumerate}
    \item Examples of context-free grammars
    \item Pushdown automata (PDA)
    \item Instantaneous descriptions of PDA's
    \item The language of a PDA
    \item Deterministic PDA
    \item From a CFG to an equivalent PDA
    \item From a PDA to an equivalent CFG
  \end{enumerate}
  
\section{Examples of Context-Free Grammars}
  \begin{enumerate}
    \item Even-length palindromes: $S \rightarrow aSa \, | \, bSb \, | \, 
    \epsilon$
    \item Odd-length palindromes: $S \rightarrow aSa \, | \, bSb \, | \, a \, | 
    \, b$
    \item Palindromes with a center marker: $S \rightarrow aSA \, | \, bSb \, | 
    \, c$
    \item Prefix notation: $E \rightarrow  + E E \, | \, * E E \, | \, a$
    \item Postfix notation: $E \rightarrow E E + \, | \, E E * \, | \, a$
    \item Balanced parentheses: $S \rightarrow ( S ) S \, | \, \epsilon$
    \item Arithmetic expressions over \texttt{id}'s and \texttt{num}'s 
    (ambiguous):
      \[
         E \rightarrow E + E \, | \, E * E \, | \, (E) \, | \, \texttt{id} \, | 
        \, \texttt{num} 
      \]
    \item Arithmetic expressions over \texttt{id}'s and \texttt{num}'s 
    (unambiguous):
      \begin{align*}
        E \rightarrow E + T \, &| \, T \\
        T \rightarrow T * F \, &| \, F \\
        F \rightarrow ( E ) \, &| \, \texttt{id} \, | \,  \texttt{num}
      \end{align*}
    \item Regular expressions over $\{ a, b \}$ (ambiguous):
      \[
         R \rightarrow R + R \, | \, RR \, | \, R^* \, | (R) \, | \, a \, | \, b 
        \, | \, \epsilon \, | \, \varphi
      \]
    \item If-then, if-then-else statements (ambiguous):
      \begin{align*}
        S &\rightarrow \text{if} \, c \, \text{then} \, S                     \\
        S &\rightarrow \text{if} \, c \, \text{then} \, S \, \text{else} \, S \\
        S &\rightarrow \text{other\_stat}
      \end{align*}
  \end{enumerate}
  
\section{Pushdown Automata}
  \begin{itemize}
    \item A pushdown automaton is an $\epsilon$-NFA with a pushdown stack 
    (last-in, first-out stack).
    \item Pushdown automata are to context-free languages as finite automata are
    to regular languages: that is to say, pushdown automata define exactly the
    context-free languages.
    \item There are seven components to a PDA $P = (Q,\Sigma,\Gamma,
    \delta,q_0,Z_0,F)$:
      \begin{enumerate}
        \item $Q$ is a finite set of states.
        \item $\Sigma$ is a finite set of input symbols (the input alphabet).
        \item $\Gamma$ is a finite set of stack symbols (the stack alphabet).
        \item $\delta$ is a transition function from $(Q \times (\Sigma \cup 
        \{ \epsilon \} \cup \Gamma)$ to subsets of $(Q \times \Gamma^*)$.
          \begin{itemize}
            \item Suppose $\delta(q, a, X)$ contains $(p, \gamma)$. Then 
            whenever $P$ is in state $q$, looking at the input symbol $a$ with 
            $X$ on top of the stack, $P$ may go into state $q$, move to the next 
            input symbol, and replace $X$ on top of the stack by the string $
            \gamma$.
            \item The second component, $a$, may be $\epsilon$ in which case $P$
            makes the move without looking at the input symbol and does not move 
            to the next input symbol.
            \item Note that if $P$ is nondeterministic, there may be more than 
            one pair in $\delta(q,a,X)$. If $P$ is nondeterministic, there may 
            be a pair in $\delta(q,a,X)$ where $a$ is a symbol in $\Sigma$ and 
            also a pair in $\delta(q,\epsilon,X)$.
          \end{itemize}
        \item $q_0$ is the start state.
        \item $Z_0$ is the start stack symbol.
        \item $F$ is the set of final (accepting states).
      \end{enumerate}
  \end{itemize}

\section{Instantaneous Descriptions of PDA's}
  \begin{itemize}
    \item We can represent a configure of the PDA $P$ above by a triple $(q,w,
    \gamma)$ where:
      \begin{itemize}
        \item $q$ is the state of the finite-state control.
        \item $w$ is the string of remaining input symbols.
        \item $\gamma$ is the string of symbols on the stack. If $\gamma = XYZ$, 
        then $X$ is on top of the stack.
      \end{itemize}
    \item Suppose $\delta(q,a,X)$ contains $(p,\alpha)$. Then to represent a 
    single move of $P$ using this transition we write
      \[ (q,aw,X\beta) |- (p,w,\alpha\beta) \]
    for all strings $w$ in $\Sigma^*$ and $\beta$ in $\Gamma^*$. Note that $a$ 
    may be $\epsilon$.
  \end{itemize}
  
\section{The Language of a PDA}
  \begin{itemize}
    \item A PDA $P = (Q, \Sigma, \Gamma, \delta, q_0, Z_0, F)$ can define a 
    language two ways.
    \item Acceptance by final state: $P$ can accept an input string $w$ by 
    reading all of it during a sequence of moves and entering a final state at 
    the end.
      \begin{itemize}
        \item Formally, we define $L(P)$, the language accepted by $P$ by final 
        state, to be the set of input strings $w$ such that $P$ can go from its 
        initial ID $(q_0,w,Z_0)$ in a sequence of zero or more moves to an 
        accepting ID of the form $(q,\epsilon,\alpha)$ where $q$ is a final 
        state and $\alpha$ is any stack string (perhaps empty).
      \end{itemize}
    \item Acceptance by empty (null) stack: $P$ can accept an input string by
    reading all of it and emptying its stack.
      \begin{itemize}
        \item Formally, we define $N(P)$, the language accepted by $P$ by empty
        stack, to be the set of input strings $w$ such that $P$ can go from its
        initial ID $(q_0,w,Z_0)$ in a sequence of zero or more moves to an
        accepting ID of the form $(q,\epsilon,\epsilon)$ for any state $q$.
        \item Note that the final state of a PDA accepting by empty stack are 
        irrelevant.
      \end{itemize}
    \item These two modes of acceptance are equivalent. That is, a language $L$ 
    has a PDA that accepts it by final state iff $L$ has a PDA that accepts it 
    by empty stack.
  \end{itemize}

\section{Deterministic Pushdown Automata}
  \begin{itemize}
    \item A PDA is deterministic (DPDA) if there is never a choice for a next 
    move in any instantaneous description.
    \item More precisely, a PDA $(Q,\Sigma,\Gamma,\delta,q_0,Z_0,F)$ is 
    deterministic if:
      \begin{enumerate}
        \item $\delta(q,a,X)$ has at most one member of any $q$ in $Q$, $a$ in 
        $\Sigma \cup \{\epsilon\}$ and $X$ in $\Gamma$.
        \item If $\delta(q,a,X)$ is nonempty for some $a$ in $\Sigma$, then 
        $\delta(q,\epsilon,X)$ must be empty.
      \end{enumerate}
    \item A language that can be recognized by a DPDA is called a deterministic 
    context-free language.
    \item A DPDA can recognize $\{ \, wcw^R \, | \, w$ is any string of $a$'s 
    and $b$'s $\}$.
    \item A PDA can recognize $\{ \, ww^R \, | w$ is any string of $a$'s and $b
    $'s $\}$, but no DPDA can recognize this language.
    \item Thus, unlike finite automata, pushdown automata with their 
    nondeterminism are strictly more powerful than deterministic pushdown 
    automata.
    \item Note that, if $L$ is a regular language, then $L$ can be recognized by 
    a DPDA.
    \item Since a DPDA can recognize the non-regular language $\{ \, wcw^R \, | 
    w$ is any string of $a$'s and $b$'s $\}$, DPDA are strictly more powerful 
    than finite automata.
  \end{itemize}

\section{From a CFG to an Equivalent PDA}
  \begin{itemize}
    \item Given a CFG $G$, we can construct a PDA $P$ such that $N(P) = L(G)$.
    \item The PDA will simulate leftmost derivations of $G$.
    \item Algorithm to construct a PDA for a CFG
      \begin{itemize}
        \item Input: A CFG $G = (V,T,Q,S)$.
        \item Output: a PDA $P$ such that $N(P) = L(G)$.
        \item Method: Let $P = (\{q\},T,V\cup{T},\delta,q,S)$ where
          \begin{enumerate}
            \item $\delta(q,\epsilon,A) = \{(q,\beta) | A \rightarrow \beta$ is 
            in $Q \}$ for each nonterminal $A$ in $V$.
            \item $\delta(q,a,a) = \{(q,\epsilon)\}$ for each terminal $a$ in 
            $T$.
          \end{enumerate}
      \end{itemize}
    \item For a given input string $w$, the PDA can simulate a leftmost 
    derivation for $w$ in $G$.
    \item We can prove that $N(P) = L(G)$ by showing that $w$ is in $N(P)$ iff 
    $w$ is in $L(G)$:
      \begin{itemize}
        \item If part: If $w$ is in $L(G)$, then there is a leftmost derivation
          \[ 
            S = Y_1 \Rightarrow Y_2 \Rightarrow \cdots Y_i \Rightarrow \cdots 
            \Rightarrow Y_n = w
          \]
        Suppose the sentential form $\gamma_i = x_i\alpha_i$ where $x_i$ is a 
        prefix of $w$ and $\alpha_i$ is a sequence of input and stack symbols 
        for $1 \leq i \leq n$. We can show by induction on $i$ that if $P$ 
        simulates this leftmost derivation by the sequence of moves
          \[ (q,w,S) |-^* (q,y_i,\alpha_i), \]
        then $x_iy_i = w$. \\ 
        This shows that if $S \overset{*}{\Rightarrow} w$, then $(q,w,S) |-^* 
        (q,\epsilon,\epsilon)$. Thus, $L(G)$ is contained in $N(P)$.
        \item Only-if part: if $(q,x,A) |-^*(q,\epsilon,\epsilon)$ then $A 
        \overset{*}{\Rightarrow} x$. \\
        We can prove this statement by induction on the number of moves made by 
        $P$. \\
        This shows that if $(q,x,A) |-^*(q,\epsilon,\epsilon)$, then $S 
        \overset{*}{\Rightarrow} w$. Thus, $N(P)$ is contained in $L(G)$.
      \end{itemize}
    \item We can now conclude $N(P) = L(G)$. Thus, every context-free language 
    can be recognized by a PDA.
  \end{itemize}

\section*{From a PDA to an equivalent CFG}
  \begin{itemize}
    \item We now show that every language recognized by a PDA can be generated 
    by a context-free grammar.
    \item Given a PDA $P$, we can construct a CFG $G$ such that $L(G) = N(P)$.
    \item The basic idea of the proof is to generate the strings that cause $P$ 
    to go from state $q$ to state $p$, popping a symbol $X$ off the stack, by a 
    nonterminal of the form $\lbrack qXp \rbrack$.
    \item Algorithm to construct a CFG for a PDA
      \begin{itemize}
        \item Input: a PDA $P = (Q,\Sigma,\Gamma,\delta,q_0,Z_0,F)$.
        \item Output: a CFG $G = (V,\Sigma,R,S)$ such that $L(G) = N(P)$.
        \item Method:
          \begin{enumerate}
            \item Let the nonterminal $S$ be the start symbol of $G$. The other 
            nonterminals in $V$ will be symbols of the form $\lbrack pXq \rbrack
            $ where $p$ and $q$ are states in $Q$, and $X$ is a stack symbol in 
            $\Gamma$.
            \item The set of productions $R$ is constructed as follows:
              \begin{itemize}
                \item For all states $p$, $R$ has the production $S \rightarrow 
                \lbrack q_0Z_0p\rbrack$.
                \item If $\delta(q,aX)$ contains $(r,Y_1Y_2\ldots{}Y_k)$, then 
                $R$ has the productions
                  \[ 
                    \lbrack qXr_k \rbrack \rightarrow a \lbrack rY_1r_1 \rbrack
                    \lbrack r_1Y_2r_2 \rbrack \ldots \lbrack r_{k-1}Y_kr_k 
                    \rbrack
                  \]
                for all lists of states $r_1, r_2, \ldots, r_k$.
              \end{itemize}
          \end{enumerate}
        \item We can prove that $\lbrack qXp \rbrack \overset{*}{\Rightarrow} w$ 
        iff $(q,w,X) |-^* (p,\epsilon,\epsilon)$.
        \item From this, we have $\lbrack q_0Z_0p \rbrack \overset{*}
        {\Rightarrow} w$ iff $(q_0,w,Z_0) |-^* (p,\epsilon,\epsilon)$, so we can 
        conclude $L(G) = N(P)$.
      \end{itemize}
    \item Sections 6 and 7 allow us to conclude that family of languages 
    generated by context-free grammars is the same as the family of languages
    recognized by pushdown automata.
    \item In summary, the regular languages are a proper subset of deterministic
    CFL's which are a proper subsets of all CFL's.
  \end{itemize}
\end{document}