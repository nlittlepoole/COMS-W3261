\documentclass[]{article}
\usepackage[all]{xy}
\usepackage{amsmath}
\usepackage{amssymb}
\usepackage{amsthm}
\usepackage{enumitem}
\usepackage{indentfirst}
\usepackage{listings}
\usepackage{multirow}
\usepackage{tikz}
\usepackage{tikz-qtree}
\usepackage{tipa}
\begin{document}

\newtheorem{thm}{Theorem}
\title{Computer Science Theory \\ COMS W3261 \\ Homework 5}
\author{Alexander Roth}
\date{2014 -- 11 -- 25}
\maketitle
\section*{Problems}
\begin{enumerate}
\item Using the following grammar for lambda calculus expressions
\[ E \rightarrow \lambda \text{var}.E\,|\,E\,E\,|\,(E)\,|\,\text{var} \]
constructs a parse tree for the expression
\[ (\lambda x.\lambda y.x\,y)\lambda z.z \]
Use the standard disambiguating conventions for lambda expressions in
constructing your parse tree.
\item[\emph{Solution}:]
\Tree [.$E$ [.$(E)$ [.$\lambda$ ] [.var [.$x$ ] ] [.. ] [.$E$ [.$\lambda$ ]
[.var [.$y$ ] ] [.. ] [.$E$ [.$E$ [.var [.$x$ ] ] ] [.$E$ [.var [.$y$ ] ] ] ] ]
] [.$E$ [.$\lambda$ ] [.var [.$z$ ] ] [.. ] [.$E$ [.var [.$z$ ] ] ] ] ]

\item Consider the lambda-calculus expression $(\lambda x.(\lambda
y.x)x)((\lambda z.z)(\lambda w.(\lambda v.v)w))$.
\begin{enumerate}
\item Identify all redexes in this expression.
\item[\emph{Solution}:] There are 4 redexes.
\begin{enumerate}
\item[1.]$(\lambda x.(\lambda y.x)x)((\lambda z.z)(\lambda w.(\lambda v.v)w))$
\item[2.]$(\lambda y.x)x$
\item[3.]$(\lambda z.z)(\lambda w.(\lambda v.v)w)$
\item[4.]$(\lambda v.v)w$
\end{enumerate}

\item Evaluate this expression using normal order evaluation.
\item[\emph{Solution}:]
\begin{align*}
(\lambda x.(\lambda y.x)x)((\lambda z.z)(\lambda w.(\lambda
v.v)w))&\underset{\beta}{\rightarrow} \lambda y.((\lambda z.z)(\lambda
w.(\lambda v.v)w))((\lambda z.z)(\lambda w.(\lambda v.v)w))\\
&\underset{\beta}{\rightarrow}(\lambda z.z)(\lambda w.(\lambda v.v)w) \\
&\underset{\beta}{\rightarrow}(\lambda w.(\lambda v.v)w) \\
&\underset{\beta}{\rightarrow}\lambda w.w
\end{align*}

\item Evaluate this expression using applicative order evaluation.
\item[\emph{Solution}:]
\begin{align*}
(\lambda x.(\lambda y.x)x)((\lambda z.z)(\lambda w.(\lambda v.v)w))
&\underset{\beta}{\rightarrow}(\lambda x.(\lambda y.x)x)((\lambda z.z)(\lambda
w.w))\\
&\underset{\beta}{\rightarrow}(\lambda x.(\lambda y.x)x)(\lambda w.w) \\
&\underset{\beta}{\rightarrow}(\lambda x.x)(\lambda w.w) \\
&\underset{\beta}{\rightarrow}\lambda w.w
\end{align*}
\end{enumerate}

\item Evaluate the lambda expression $(\lambda x.(\lambda y.(x(\lambda
x.x\,y))))y$. Describe all the steps in your evaluation.
\item[\emph{Solution}:]
\begin{align}
(\lambda x.(\lambda y.(x(\lambda x.x\,y))))y &\underset{\alpha}{\rightarrow}
(\lambda z.(\lambda y.(z(\lambda x.x\,y))))y \\
&\underset{\alpha}{\rightarrow}(\lambda z.(\lambda q.(z(\lambda x.x\,y))))y \\
&\underset{\beta}{\rightarrow}(\lambda q.(y(\lambda x.x y)))
\end{align}
\begin{enumerate}
\item[(1)] Alpha reduction to remove ambiguity between bound variables and free
variables. The outermost $\lambda x$ is converted to $\lambda z$ along with the
bound $x$ variable.
\item[(2)] Alpha reduction to remove ambiguity between bound variables and free
variables. $\lambda y$ is converted to $\lambda q$ to avoid ambiguity with the
free $y$ on the outside of the equation.
\item[(3)] Beta reduction. Substitute the $y$ argument for any $z$'s in the
equation. We cannot reduce anymore from here so we are left with the equation
$\lambda q.(y(\lambda x.x\,y))$.
\end{enumerate}
\item Let $G$ be the lambda abstraction
\[ G = (\lambda f.\lambda x.f(f\,x)) \]
Evaluate the lambda expression $G\,G$.
\item[\emph{Solution}:]
\begin{align*}
GG &= (\lambda f.\lambda x.f(f\,x))G\\
&\underset{\beta}{\rightarrow} \lambda x.G(G\,x) \\
&\underset{\alpha}{\rightarrow} \lambda x.G((\lambda f.\lambda
x^\prime.f(f\,x^\prime)) x) \\
&\underset{\beta}{\rightarrow} \lambda x.G(\lambda x^\prime.x(x\,x^\prime)) \\
&\underset{\alpha}{\rightarrow} \lambda x.(\lambda f.\lambda y.f(f\,y))(\lambda
x^\prime.x(x\,x^\prime)) \\
&\underset{\beta}{\rightarrow} \lambda x.\lambda y.(\lambda
x^\prime.x(x\,x^\prime))((\lambda x^\prime.x(x\,x^\prime))\,y) \\
&\underset{\beta}{\rightarrow} \lambda x.\lambda y.(\lambda
x^\prime.x(x\,x^\prime)) (x(x\,y)) \\
&\underset{\beta}{\rightarrow} \lambda x.\lambda y.x(x(x(x(x\,y))))\\
&\underset{\alpha}{\rightarrow} \lambda f.\lambda y.f(f(f(f\,y)))) \\
&\underset{\alpha}{\rightarrow} \lambda f.\lambda x.f(f(f(f\,x))))
\end{align*}

\item Let add, one, and two be the following lambda expressions:
\begin{align*}
\text{add} &= \lambda m.\lambda n.\lambda f.\lambda x.m\,f(n\,f\,x) \\
\text{one} &= \lambda f.\lambda x.f\,x \\
\text{two} &= \lambda f.\lambda x.f(f\,x)
\end{align*}
Evaluate (add one two).
\item[\emph{Solution}:]
\begin{align*}
(\textrm{add one two}) &=((\lambda m.\lambda n.\lambda f.\lambda x.m
f(n\,f\,x))\textrm{one}\,\,\text{two})\\
&\underset{\beta}{\rightarrow}(\lambda n.\lambda f.\lambda
x.\textrm{one}\,f(n\,f\,x))\,\textrm{two} \\
&\underset{\beta}{\rightarrow}(\lambda f.\lambda
x.\textrm{one}\,f(\textrm{two}\,f\,x)) \\
&\underset{\alpha}{\rightarrow}(\lambda f.\lambda x.(\lambda f^\prime.\lambda
x\prime.f^\prime\,x^\prime)f(\textrm{two}\,f\,x)) \\
&\underset{\beta}{\rightarrow} (\lambda f.\lambda x.(\lambda
x^\prime.f\,x^\prime)(\textrm{two}\,f\,x)) \\
&\underset{\beta}{\rightarrow} \lambda f.\lambda x.f\,(\textrm{two}\,f\,x) \\
&\underset{\alpha}{\rightarrow} \lambda f.\lambda x.f\,((\lambda
f^\prime.\lambda x^\prime.f^\prime(f^\prime\,x^\prime))\,f\,x) \\
&\underset{\beta}{\rightarrow} \lambda f.\lambda x.f (\lambda
x^\prime.f(f\,x^\prime) x) \\
&\underset{\beta}{\rightarrow} \lambda f.\lambda x.f(f(f\,x))
\end{align*}
\end{enumerate}

\end{document}
