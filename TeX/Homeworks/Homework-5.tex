\documentclass[]{article}
\usepackage[all]{xy}
\usepackage{amsmath}
\usepackage{amssymb}
\usepackage{amsthm}
\usepackage{enumitem}
\usepackage{indentfirst}
\usepackage{listings}
\usepackage{multirow}
\usepackage{tikz}
\usepackage{tikz-qtree}
\usepackage{tipa}
\begin{document}

\newtheorem{thm}{Theorem}
\title{Computer Science Theory \\ COMS W3261 \\ Homework 5}
\author{Alexander Roth}
\date{2014 -- 11 -- 25}
\maketitle
\section*{Problems}
\begin{enumerate}
\item Using the following grammar for lambda calculus expressions
\[ E \rightarrow \lambda \text{var}.E\,|\,E\,E\,|\,(E)\,|\,\text{var} \]
constructs a parse tree for the expression
\[ (\lambda x.\lambda y.x\,y)\lambda z.z \]
Use the standard disambiguating conventions for lambda expressions in constructing your parse tree.
\item[\emph{Solution}:]

\item Consider the lambda-calculus expression $(\lambda x.(\lambda y.x)x)((\lambda z.z)(\lambda w.(\lambda v.v)w))$.
\begin{enumerate}
\item Identify all redexes in this expression.
\item[\emph{Solution}:]

\item Evaluate this expression using normal order evaluation.
\item[\emph{Solution}:]

\item Evaluate this expression using applicative order evaluation.
\item[\emph{Solution}:]
\end{enumerate}


\item Evaluate the lambda expression $(\lambda x.(\lambda y.(x(\lambda x.x\,y))))y$. Describe all the steps in your evaluation.

\item Let $G$ be the lambda abstraction
\[ G = (\lambda f.\lambda x.f(f\,x)) \]
Evaluate the lambda expression $G\,G$.
\item[\emph{Solution}:]

\item Let add, one, and two be the following lambda expressions:
\begin{align*}
\text{add} &= \lambda m.\lambda n.\lambda f.\lambda x.m\,f(n\,f\,x) \\
\text{one} &= \lambda f.\lambda x.f\,x \\
\text{two} &= \lambda f.\lambda x.f(f\,x)
\end{align*}
Evaluate (add one two).
\item[\emph{Solution}:]
\end{enumerate}

\end{document}
