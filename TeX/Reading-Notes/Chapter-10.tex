\documentclass[]{article}
\usepackage{amsmath}
\usepackage{amssymb}
\usepackage{amsthm}
\usepackage{bm}
\usepackage[T1]{fontenc}
\usepackage{indentfirst}
\usepackage{listings}
\usepackage{tikz}
\usepackage{tikz-qtree}
\usepackage{tipa}
\usetikzlibrary{arrows,automata}
\begin{document}

\title{COMS W3261 \\ Computer Science Theory \\ Chapter 10 Notes}
\author{Alexander Roth}
\date{2014--11--08}
\maketitle
\theoremstyle{definition}
\newtheorem{thm}{Theorem}
\section*{Intractable Problems}
Recall
\begin{itemize}
\item The problems solvable in polynomial time on a typical computer are exactly
the same as the problems solvable in polynomial time on a Turing machine.
\end{itemize}

\section*{The Classes $\mathcal{P}$ and $\mathcal{NP}$}
\subsection*{Problems Solvable in Polynomial Time}
A Turing machine $M$ is said to be of \emph{time complexity} $T(n)$ if whenever
$M$ is given an input $w$ of length $n$, $M$ halts after making at most $T(n)$
moves, regardless of whether or not $M$ accepts. A language $L$ is in class
$\mathcal{P}$ if there is some polynomial $T(n)$ such that $L = L(M)$ for some
deterministic TM $M$ of time complexity $T(n)$.

\subsection*{An Example: Kruskal's Algorithm}
There is a well-known ``greedy'' algorithm, called \emph{Kruskal's Algorithm},
for finding a minimum-weight spanning tree.
\begin{enumerate}
\item Maintain for each node the \emph{connected component} in which the node
appears, using whatever edges of the tree have been selected so far. Initially,
no edges are selected, so every node is then in a connected component by itself.
\item Consider the lowest-weight edge that has not yet been considered; break
ties any way you like. If this edge connects two nodes that are currently in
different connected components then:
\begin{enumerate}
\item Select that edge for the spanning tree, and
\item Merge the two connected components involved, by changing the component
number of all nodes in one of the two components to be the same as the component
number of the other.
\end{enumerate}
If, on the other hand, the selected edge connects two nodes of the same
component, then this edge does not belong in the spanning tree; it would create
a cycle.
\item Continue considering edges until either all edges have been considered, or
the number of edges selected for the spanning tree is one less than the number
of nodes. Note that in the latter case, all nodes must be in one connected
component, and we can stop considering edges.
\end{enumerate}

In the theory of intractability, we generally want to argue that a problem is
hard, not easy, and the fact that a yes-no version of a problem is hard implies
that a more standard version, where a full answer must be computed, is also
hard.

Problem elements must be encoded suitably to work with a Turing machine. The
effect of this requirement is that inputs to Turing machines are generally
slightly longer than the intuitive ``size'' of the input. However, there are two
reasons why the difference is not significant:
\begin{enumerate}
\item The difference between the size as a TM input string and as an informal
problem input is never more than a small factor, usually the logarithm of the
input size.
\item The length of a string representing the input is actually a more accurate
measure of the number of bytes a real computer has to read to get its input.
\end{enumerate}
\end{document}
