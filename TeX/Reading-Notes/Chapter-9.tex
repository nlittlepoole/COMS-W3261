\documentclass[]{article}
\usepackage{amsmath}
\usepackage{amssymb}
\usepackage{amsthm}
\usepackage{bm}
\usepackage[T1]{fontenc}
\usepackage{indentfirst}
\usepackage{listings}
\usepackage{tikz}
\usepackage{tikz-qtree}
\usepackage{tipa}
\usetikzlibrary{arrows,automata}
\begin{document}

\title{COMS W3261 \\ Computer Science Theory \\ Chapter 9 Notes}
\author{Alexander Roth}
\date{2014--11--02}
\maketitle
\theoremstyle{definition}
\newtheorem{thm}{Theorem}
\section*{Undecidability}
\section*{A Language That Is Not Recursively Enumerable}
Recall that a language $L$ is \emph{recursively enumerable} if $L = L(M)$ for
some TM $M$.

Our long-range goal is to prove undecidable the language consisting of pairs
$(M,w)$ such that:
\begin{enumerate}
\item $M$ is a Turing machine (suitably coded, in binary) with input alphabet
$\{0,1\}$,
\item $w$ is a string of $0$'s and $1$'s, and
\item $M$ accepts input $w$.
\end{enumerate}
If this problem with inputs restricted to the binary alphabet is undecidable,
then surely the more general problem, where TM's may have any alphabet, is
undecidable.

We shall show that the language $L_d$, the ``diagonalization language,'' which
consists of all those strings $w$ such that the TM represented by $w$ does not
accept the input $w$, has no Turing machine that accepts it. Shwoing there is no
Turing machine at all for a language is showing that it has no algorithm or TM
that will always halt.

\subsection*{Enumerating the Binary Strings}
If $w$ is a binary string, treat $1w$ as a binary integer $i$. Then we shall
call $w$ the $i$th string. That is, $\epsilon$, is the first string, 0 is the
second, 1 the third, 00 the fourth, 01 the fifth, and so on. Equivalent.y,
strings are ordered by length and strings of equal length are ordered
lexicographically. Hereafter, we shall refer to the $i$th string as $w_i$.

\subsection*{Codes for Turing Machines}
To represent a TM $M (Q, \{0,1\}, \Gamma, \delta, q_1, B, F)$ as a binary
string, we must first assign integers to the states, tape symbols, and
directions $L$ and $R$.
\begin{itemize}
\item We shall assume the states are $q_1,q_2,\ldots,q_r$ for some $r$. The
start state will always be $q_1$, and $q_2$ will be the only accepting state.
\item We shall assume the tape symbols are $X_1,X_2,\ldots,X_s$ for some $s$.
$X_1$ always will be the symbol 0, $X_2$ will be 1, and $X_3$ will be $B$, the
blank.
\item We shall refer to direction $L$ as $D_1$ and direction $R$ as $D_2$.
\end{itemize}
Since each TM $M$ can have integers assigned to its states and tape symbols in
many different orders, there will be more than one encoding of the typical TM.

Suppose one transition rule is $\delta(q_i, X_j) = (q_k, X_l, D_m)$, for some
integers $i$, $j$, $k$, $l$, and $m$.  We shall code this rule by the string
$0^i10^j10^k10^l10^m$. Notice that, since all of $i$, $j$, $k$, $l$, and $m$ are
at least one, there are no occurrences of two or more consecutive $1$'s within
the code for a single transition.

A code for the entire TM $M$ consists of all the codes for the transitions, in
some order, separated by pairs of $1$'s:
\[ C_111C_211\cdots{C_{n-1}}11C_n \]
where each of the $C$'s is the code for one transition of $M$.

\subsection*{The Diagonalization Language}
If $w_i$ is not a valid TM code, we shall take $M_i$ to be the TM with one state
and no transitions. That is, for these values of $i$, $M_i$ is a Turing machine
that immediately halts on any input. Thus, $L(M_i)$ is $\emptyset$ if $w_i$
fails to be a valid TM code.

That is,
\begin{itemize}
\item The language $L_d$, the \emph{diagonalization language}, is the set of
strings $w_i$ such that $w_i$ is not in $L(M_i)$.
\end{itemize}
That is, $L_d$ consists of all strings $w$ such that the TM $M$ whose code is
$w$ does not accept when given $w$ as input.

$L_d$ can be represented in tabular form. The $i$th row can be thought of as the
\emph{characteristic vector} for the language $L(M_i)$; that is, the 1's in this
row indicate the strings that are members of this language.

The diagonal values tell whether $M_i$ accepts $w_i$. To construct $L_d$, we
complement the diagonal.

The trick of complementing the diagonal to construct the characteristic vector
of a language that cannot be the language that appears in any row is called
\emph{diagonalization}. It works because the complement of the diagonal is
itself a characteristic vector describing membership in some language, namely
$L_d$.

\subsection*{Proof that $L_d$ is Not Recursively Enumerable}
There is no Turing machine that accepts language $L_d$.
\begin{thm}
$L_d$ is not a recursively enumerable language. That is, there is no Turing
machine that accepts $L_d$.
\end{thm}

\section*{An Undecidable Problem That Is RE}
\subsection*{Recursive Languages}
We call a language $L$ \emph{recursive} if $L = L(M)$ for some Turing machine
$M$ such that:
\begin{enumerate}
\item If $w$ is in $L$, then $M$ accepts (and therefore halts).
\item If $w$ is not in $L$, then $M$ eventually halts, although it never enters
an accepting state.
\end{enumerate}
A TM of this type corresponds to our notion of an ``algorithm,'' a well-defined
sequence of steps that always finishes and produces an answer. If we think of
the language $L$ as a ``problem,'' then problem $L$ is called \emph{decidable}
if it is a recursive language, and it is called \emph{undecidable} if it is not
a recursive language.

Dividing problems or languages between the decidable and those that are
undecidable is often more important than the division between the recursively
enumerable languages and the non-recursively-enumberal languages. The
relationship among the three classes of languages is:
\begin{enumerate}
\item The recursive languages.
\item The languages that are recursively enumerable but not recursive.
\item The non-recursively-enumberable (\emph{non-RE}) languages.
\end{enumerate}

\subsection*{Complements of Recursive and RE languages}
If a language $L$ is RE, but $\overline{L}$, the complement of $L$, is not RE,
then we know $L$ cannot be recursive. For if $L$ were recursive, then
$\overline{L}$ would also be recursive and thus surely RE.
\begin{thm}
If $L$ is a recursive language, so is $\overline{L}$.
\end{thm}
\begin{thm}
If both a language $L$ and its complement are RE, then $L$ is recursive. Note
that then by Theorem 2, $\overline{L}$ is recursive as well.
\end{thm}
Of the nine possible ways to place a language $L$ and its complement
$\overline{L}$, only the following four are possible:
\begin{enumerate}
\item Both $L$ and $\overline{L}$ are recursive.
\item Neither $L$ nor $\overline{L}$ is RE.
\item $L$ is RE but no recursive, and $\overline{L}$ is not RE.
\item $\overline{L}$ is RE but not recursive, and $L$ is not RE.
\end{enumerate}

\subsection*{The Universal Language}
We define $L_u$, the \emph{universal language}, to be the set of binary strings
that encode a pair $(M, w)$, where $M$ is a TM with the binary input alphabet,
and $w$ is a string in $(\bm{0} + \bm{1})^*$, such that $w$ is in $L(M)$. That
is, $L_u$ is the set of strings representing a TM and an input accepted by that
TM. There is a TM $U$, often called the \emph{universal Turing machine}, such
that $L_u = L(U)$. Since the input to $U$ is a binary string, $U$ is in fact
some $M_j$ in the list of binary-input Turing machines.

$U$ is a multitape Turing machine. In the case of $U$, the transitions of $M$
are stored initially on the first tape, along with the string $w$. A second tape
is used to hold the simulated tape of $M$, using the same format as for the code
of $M$. That is, tape symbols $X_i$ of $M$ will be represented by $0^i$, and
tape symbols will be separated by single $1$'s. The third tape of $U$ holds the
state of $M$, with state $q_i$ represented by $i$ $0$'s.

The operation of $U$ can be summarized as follows:
\begin{enumerate}
\item Examine the input to make sure that the code for $M$ is a legitimate code
for some TM. If not, $U$ halts without accepting.
\item Initialize the second tape to contain the input $w$, in its encoded form.
That is, for each $0$ of $w$, place 10 on the second tape, and for each 1 of
$w$, place 100 there. Note that the blanks on the simulated tape of $M$, which
are represented by 1000, will not actually appear on that tape; all cells beyond
those used for $w$ will hold the blank of $U$.
\item Place 0, the start state of $M$, on the third tape, and move the head of
$U$'s second tape to the first simulated cell.
\item To simulate a move of $M$, $U$ searches on its first tape for a transition
$0^i10^j10^k10^l10^m$, such that $0^i$ is the state on tape 3, and $O^j$ is the
tape symbol of $M$ that begins at the position on tape 2 scanned by $U$.
\item If $M$ has no transition that matches the simulated state and tape symbol,
no transition is found. Thus, $M$ halts in the simulated configuration, and $U$
must do likewise.
\item If $M$ enters its accepting state, then $U$ accepts.
\end{enumerate}
In this manner, $U$ simulates $M$ on $w$. $U$ accepts the coded pair $(M, w)$ if
and only if $M$ accepts $w$.

\subsection*{Undecidability of the Universal Language}
\begin{thm}
$L_u$ is RE but not recursive.
\end{thm}
Knowing that $L_u$ is undecidable (i.e., not a recursive language) is in many
ways more value that knowing that $L_d$ is not RE. The reduction of $L_u$ to
another problem $P$ can be used to show there is no algorithm to solve $P$,
regardless of whether or not $P$ is not RE. However, reduction of $L_d$ to $P$
is only possible if $P$ is not RE, so $L_d$ cannot be used to show
undecidability for those problems that are RE but not recursive. If we want to
show a problem not to be RE, then only $L_d$ can be used; $L_u$ is useless since
it \emph{is} RE.

\subsubsection*{THe Halting Problem}
We could define $H(M)$ for TM $M$ to be the set of inputs $w$ such that $M$
halts given input $w$, regardless of whether or not $M$ accepts $w$. Then, the
\emph{halting problem} is the set of pairs $(M, w)$ such that $w$ is in $H(M)$.
This problem/language is another example of one that is RE but not recursive.

\section*{Post's Correspondence Problem}
\subsection*{Definition of Post's Correspondence Problem}
An instance of \emph{Post's Correspondence Problem} (PCP) consists of two lists
of strings over some alphabet $\Sigma$; the two lists must be of equal length:
$A$ and $B$, where $A = w_1, w_2,\ldots,w_k$ and $B = x_1,x_2,\ldots,x_k$, for
some integer $k$. For each $i$, the pair $(w_i,x_i)$ is said to be a
\emph{corresponding} pair.

The instance of PCP \emph{has a solution}, if there is a sequence of one or more
integers $i_1, i_2,\ldots,i_m$ that, when interpreted as indexes for strings in
the $A$ and $B$ lists, yield the same string. That is,
$w_{i_1},w_{i_2},\ldots,w_{i_m} = x_{i_1},x_{i_2},\ldots,x_{i_m}$. Thus,
$i_1,i_2,\ldots,i_m$ is a \emph{solution} to this instance of PCP.

\subsubsection*{Partial Solutions}
The possible \emph{partial soultions} of PCP instances are sequences of indexes
$i_1, i_2, \ldots, i_r$ such that one of $w_{i_1}, w_{i_2}, \cdots, w_{i_r}$ and
$x_{i_1}, x_{i_2}, \cdots, x_{i_r}$ is a prefix of the other, although the two
strings are not equal. Notice that if a sequence of integers is a solution, then
every prefix of that sequence must be a partial solution. Thus, understanding
what the partial solutions are allows us to argue about what solutions there
might be.

However, because PCP is undecidable, there is no algorithm to compute all the
partial solutions. Thus, there can be an infinite number of them.

\subsection*{The ``Modified'' PCP}
In the modified PCP, there is the additional requirement on a solution that the
first pair on the $A$ and $B$ lists must be the first pair in the solution. An
instance of MCPCP is two lists $A = w_1, w_2, \ldots, w_k$ and $B =
x_1,x_2,\ldots,x_k$, and a solution is a list of 0 or more integers
$i_1,i_2,\ldots,i_m$ such that
\[ w_1w_{i_1}w_{i_2}\cdots w_{i_m} = x_1x_{i_1}x_{i_2}\cdots x_{i_m} \]

Notice that the pair $(w_1,x_1)$ is forced to be at the beginning of the two
strings, even though the index 1 is not mentioned at the front of the list that
is the solution.

AN important step in showing PCP is undecidable is reducing MPCP to PCP.

We are given an instance of MPCP with lists $A = w_1,w_2,\ldots,w_k$ and $B =
x_1,x_2,\ldots,x_k$. We assume $*$ and \$ are symbols not present in the
alphabet $\Sigma$ of this MPCP instance. We construct a PCP instance $C =
y_0,y_1,\ldots,y_{k+1}$ and $D = z_0,z_1,\ldots,z_{k+1}$ as follows:
\begin{enumerate}
\item For $i = 1,2,\ldots,k$, let $y_i$ be $w_i$ with a $*$ after each symbol of
$w_i$, and let $z_i$ be $x_i$ with a $*$ before each symbol of $x_i$.
\item $y_0 = *y_1$ and $z_0 = z_1$.
\item $y_{k+1} = \$$ and $z_{k+1} = *\$$.
\end{enumerate}

\begin{thm}
MPCP reduces to PCP.
\end{thm}

\subsection*{Completion of the Proof of PCP Undecidability}
Given a pair $(M, w)$, we construct an instance $(A, B)$ of MPCP such that TM
$M$ accepts input $w$ if and only if $(A, B)$ has a solution.

The essential idea is that MPCP instance $(A, B)$ simulates, in its partial
solutions, the computation of $M$ on input $w$. Partial solutions will consist
of strings that are prefixes of the sequence of ID's of $M$:
$\#\alpha_1\#\alpha_2\cdots$, where $\alpha_1$ is the initial ID of $M$ with
input $w$, and $\alpha_i \vdash \alpha_{i+1}$ for all $i$. The string from $B$
list will always be one ID ahead of the string from the $A$ list, unless $M$
enters an accepting state. In that case, there will be pairs to use that will
allow the $A$ list to ``catch up'' to the $B$ list and eventually produce a
solution.

We may assume our TM never prints a blank, and never moves left from its initial
head position. In that case, an ID of the Turing machine will always be a string
of the form $\alpha{q}\beta$, where $\alpha$ and $\beta$ are strings of nonblank
tape symbols, and $q$ is a state. The symbols of $\alpha$ and $\beta$ will
correspond exactly to the contents of the cells that held the input, plus any
cells to the right that the head has previously visited.

Let $M = (Q,\Sigma,\Gamma,\delta,q_0,B,F)$ be a TM described above and let $w$
in $\Sigma^*$ be an input string.
\begin{enumerate}
\item The first pair is $(\#, \#q_0w\#)$. This pair, which must start any
solution according to the rules of MPCP, begins the simulation of $M$ on input
$w$.
\item Tape symbols and the separator $\#$ can be appended to both lists. In
effect, choice of these pairs lets us extend the $A$ string to match the $B$
string, and at the same time copy parts of the previous ID to the end of the $B$
string.
\item To simulate a move of $M$, we have certain pairs that reflect those moves.
These pairs help extend the $B$ string to add the next ID, by extending the $A$
string to match the $B$ string. However, these pairs use the state to determine
the change in the current ID that is needed to produce the next ID. These
changes are reflected in the ID being constructed at the end of the $B$ string.
\item If the ID at the end of the $B$ string has an accepting state, then we
need to allow the partial solution to become a complete solution. We do so by
extending with ``ID's'' that are not really ID's of $M$, but represent what
would happen if the accepting state were allowed to consume all the tape symbols
to either side of it.
\item Finally, once the accepting state has consumed all tape symbols, it stands
alone as the last ID on the $B$ string. That is, the \emph{remaineder} of the
two strings is $q\#$.
\end{enumerate}

\begin{thm}
Post's Correspondence Problem is undecidable.
\end{thm}

\section*{Other Undecidable Problems}
The principal technique is reducing PCP to the problem we wish to prove
undecidable.
\subsection*{Problems About Programs}
We can write a program, in any conventional language, that takes as input an
instance of PCP and searches for solutions some systematic manner, e.g., in
order of the \emph{length} of potential solutions. Since PCP allows arbitrary
alphabets, we should encode the symbols of its alphabet in binary or some other
fixed alphabet.

We can have our program do any particular thing we want when and if it finds a
solution. Otherwise, the program will never perform that particular action.
Thus, it is undecidable whether a program makes a nontrivial action. Any
nontrivial property that involves what the program does must be undecidable.

\subsection*{Undecidability of Ambiguity for CFG's}
Let the PCP instance consist of lists $A = w_1,w_2,\ldots,w_k$ and $B =
x_1,x_2,\ldots,x_k$. For list $A$ we shall construct a CFG with $A$ as the only
variable. The terminals are all the symbols of the alphabet $\Sigma$ used for
this PCP instance, plus a distinct set of \emph{index symbols}
$a_1,a_2,\ldots,a_k$ that represents the choices of pairs of strings in a
solution to the PCP instance. That is, the index symbol $a_i$ represents the
choice of $w_i$ from the $A$ list or $x_i$ from the $B$ list. The productions
for the CFG for the $A$ list are:
\begin{align*}
A \quad \rightarrow \quad & w_1Aa_1\,|\,w_2Aa_2\,|\,\cdots\,|\,w_kAa_k\,| \\
                          & w_1a_1 \,|\,w_2a_2 \,|\,\cdots\,|\,w_ka_k
\end{align*}
We shall call this grammar $G_A$ and its language $L_A$, or \emph{the language
for the list A}.

Notice that the terminal strings derived by $G_A$ are all those of the form
$w_{i_1}w_{i_2}\cdots w_{i_m}a_{i_m}\cdots{a_{i_2}}a_{i_1}$ for some $m \geq 1$
and list of integers $i_1,i_2,\ldots,i_m$; each integer is in the range 1 to
$k$. The sentential forms of $G_A$ all have a single $A$ between the strings and
the index symbols, until we use one of the last group of $k$ productions, none
of which has an $A$ in the body.

Observer also that any terminal string derivable from $A$ in $G_A$ has a unique
derivation.

We can develop another grammar $G_B$ from the list $B = x_1,x_2,\ldots,x_k$ as
follows:
\begin{align*}
B \quad \rightarrow \quad & x_1Ba_1\,|\,x_2Ba_2\,|\,\cdots\,|\,x_kBa_k\,| \\
                          & x_1a_1 \,|\,x_2a_2 \,|\,\cdots\,|\,x_ka_k
\end{align*}
The language of this grammar will be referred to as $L_B$. The same observations
of $L_A$ hold on $L_B$.

Finally, we combine the languages and grammars of the two lists to form a
grammar $G_{AB}$ for the entire PCP instance. $G_{AB}$ consists of:
\begin{enumerate}
\item Variable $A$, $B$, and $S$; the latter is the start symbol.
\item Productions $S \rightarrow A\,|\,B$.
\item All the productions of $G_A$.
\item All the productions of $G_B$.
\end{enumerate}
$G_{AB}$ is ambiguous if and only if the instance $(A, B)$ of PCP has a
solution.

\subsection*{The Complement of a List Language}
\begin{thm}
If $L_A$ is the language of list $A$, then $\overline{L_A}$ is a context-free
language.
\end{thm}
\begin{thm}
Let $G_1$ and $G_2$ be context-free grammars, and let $R$ be a regular
expression. Then the following are undecidable:
\begin{enumerate}
\item Is $L(G_1) \cap L(G_2) = \emptyset$?
\item Is $L(G_1) = L(G_2)$?
\item Is $L(G_1) = L(R)$?
\item Is $L(G_1) = T^*$ for some alphabet $T$?
\item Is $L(G_1) \subseteq L(G_2)$?
\item Is $L(R) \subseteq L(G_1)$?
\end{enumerate}
\end{thm}
\end{document}
