\documentclass[]{article}
\usepackage{amsmath}
\usepackage{amssymb}
\usepackage{amsthm}
\usepackage{bm}
\usepackage[T1]{fontenc}
\usepackage{indentfirst}
\usepackage{listings}
\usepackage{tikz}
\usepackage{tikz-qtree}
\usepackage{tipa}
\usetikzlibrary{arrows,automata}
\begin{document}

\title{COMS W3261 \\ Computer Science Theory \\ Chapter 11 Notes}
\author{Alexander Roth}
\date{2014--11--17}
\maketitle
\theoremstyle{definition}
\newtheorem{thm}{Theorem}

\section*{Additional Classes of Problems}
\section*{Complements of Languages in $\mathcal{NP}$}
The class of languages $\mathcal{P}$ is closed under complementation. It is not
known whether $\mathcal{NP}$ is closed under complementation.

\subsection*{The Class of Lagnuages Co-$\mathcal{NP}$}
Co-$\mathcal{NP}$ is the set of languages whose complements are in
$\mathcal{NP}$Every language $\mathcal{P}$ has its complement also in
$\mathcal{P}$, and therefore in $\mathcal{NP}$. We believe that none of the NP-
Complete problem have their complements in $\mathcal{NP}$, and therefore no NP-
Complete problem is in Co-$\mathcal{NP}$. Likewise, we believe the complements
of NP-Complete problems, which are by definition in Co-$\mathcal{NP}$, are not
in $\mathcal{NP}$.

\subsection*{NP-Complete Problems and Co-$\mathcal{NP}$}
Let us assume that $\mathcal{P} \neq \mathcal{NP}$. We could have $\mathcal{NP}$
and co-$\mathcal{NP}$ are equal, but larger than $\mathcal{P}$.
\begin{thm}
$\mathcal{NP} =$ Co-$\mathcal{NP}$ if an only if there is some NP-complete
problem whose complement is in $\mathcal{NP}$.
\end{thm}
\end{document}
