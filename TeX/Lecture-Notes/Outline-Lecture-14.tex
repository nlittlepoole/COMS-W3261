\documentclass[]{article}
\usepackage{amsmath}
\usepackage{amssymb}
\usepackage{amsthm}
\usepackage{listings}
\usepackage{multirow}
\usepackage{tikz}
\usepackage{tikz-qtree}
\usepackage{tipa}
\usetikzlibrary{arrows,automata}
\begin{document}
\newcommand*{\xml}[1]{\texttt{<#1>}}
\theoremstyle{definition}
\newtheorem{thm}{Theorem}
\title{COMS W3261 \\ Computer Science Theory \\ Lecture 14\\ Algorithms and the
Church-Turing Thesis}
\author{Alexander Roth}
\date{2014 -- 10 -- 20}
\maketitle
\section*{Outline}
  \begin{enumerate}
    \item Midterm review
    \item Definition of an algorithm
    \item The Church-Turing thesis
    \item The diagonalization language $L_d$ is not RE
    \item Reducing one problem to another
  \end{enumerate}
\section{Definition of Algorithm}
  \begin{itemize}
    \item Surprisingly, there is no universally agreed-upon definition for the
    term ``algorithm''. Informally, we can think of an algorithm as a collection
    of well-defined instructions for carrying out some task.
    \item In \emph{The Art of Computer Programming}, Donald Knuth states that an
    algorithm should have five properties.
      \begin{enumerate}
        \item Finiteness: An algorithm must always terminate after a finite
        number of steps.
        \item Definiteness: Each step of an algorithm must be precisely defined.
        \item Input: An algorithm has zero or more inputs.
        \item Output: An algorithm has one or more outputs, quantities which
        have a specified relation to the inputs.
        \item Effectiveness: All of the operations to be performed in an
        algorithm can be done exactly and in a finite length of time.
      \end{enumerate}
    \item In this course we will use a Turing machine that halts on all inputs
    as the definition of an algorithm. The term decider is sometimes used for 
    such a Turing machine.
      \begin{itemize}
        \item A language $L$ that can be recognized by an algorithm is said to
        be recursive.
        \item If a language $L$ is recursive, we say $L$ is decidable.
        \item If a language $L$ is not recursive, we say $L$ is undecidable.
      \end{itemize}
    \item In general, a Turing machine need not halt all inputs. An input on
    which a Turing machine never halts is not in the language defined by the 
    Turing machine.
      \begin{itemize}
        \item A language $L$ that can be recognized by a Turing machine is said
        to be recursively enumerable.
        \item The term Turing-recognizable language is sometimes used for a
        recursively enumerable language.
        \item Note that a language may be undecidable because it is not
        recursive but is recursively enumerable or because it is not 
        recursively enumerable.
      \end{itemize}
  \end{itemize}
\section{The Church-Turing Thesis}
  \begin{itemize}
    \item A Turing machine can compute a function from an input to an output by
    reading the input, making a sequence of moves, and then halting, leaving
    only the output of the function on the tape.
    \item A recursive function is one that can be computed by a Turing machine
    that halts on all inputs.
    \item A partial-recursive function is one that can be computed by a Turing
    machine that need not halt on all inputs. The output of the function on an
    input for which the Turing machine does not halt is said to be undefined.
    \item The Church-Turing thesis says that any general way to compute will
    allow us to compute only the partial-recursive functions. The Church-Turing 
    thesis is unprovable because there is no precise definition for ``any
    general way to compute.''
    \item An informal way of expressing the Church-Turing thesis is that any
    function that can be effectively computed can be computed by a Turing
    machine.
  \end{itemize}
\section{The Diagonalization Language $L_d$ is not Recursively Enumerable}
  \begin{itemize}
    \item We can enumerate all binary strings.
    \item We can enumerate all Turing machines.
    \item We define $L_d$, the diagonalization language, as follows:
      \begin{enumerate}
        \item Let $w_1$, $w_2$, $w_3$, $\ldots$ be an enumeration of all binary
        strings.
        \item Let $M_1$, $M_2$, $M_3$, $\ldots$ be an enumeration of all Turing
        machines.
        \item Let $L_d = \{\,w_i\,|\,w_i$ is not in $L(M_i)\,\}$.
      \end{enumerate}
    \item Theorem: $L_d$ is not a recursively enumerable language.
    \item Proof:
      \begin{itemize}
        \item Suppose $L_d = L(M_i)$ for some TM $M_i$.
        \item This gives rise to a contradiction. Consider what $M_i$ will do on
        the input $w_i$.
          \begin{itemize}
            \item If $M_i$ accepts $w_i$, then by definition $w_i$ cannot be in
            $L_d$.
            \item If $M_i$ does not accept $w_i$, then by definition $w_i$ is in
            $L_d$.
          \end{itemize}
        \item Since $w_i$ can neither be in $L_d$ nor not be in $L_d$, we must
        conclude there is no Turing machine that can define $L_d$.
      \end{itemize}
  \end{itemize}
\section{Reducing One Problem to Another}
  \begin{itemize}
    \item If we have an algorithm to convert instance of a problem $P_1$ to
    instances of a problem $P_2$ that have the same answer, then we say that
    $P_1$ reduces to $P_2$.
    \item A reduction from $P_1$ to $P_2$ must turn every instance of $P_1$ with
    a yes answer to an instance of $P_2$ with a yes answer, and every instance 
    of $P_1$ with a no answer to an instance of $P_2$ with a no answer.
    \item We will frequently use this technique to show that problem $P_2$ is as
    hard as problem $P_1$.
    \item The direction of the reduction is important.
    \item For example, if there is a reduction from $P_1$ to $P_2$ and if $P_1$
    is not recursive, then $P_2$ cannot be recursive.
    \item Similarly, if there is a reduction from $P_1$ to $P_2$ and if $P_1$ is
    not recursively enumerable, then $P_2$ cannot be recursively enumerable.
  \end{itemize}
\end{document}
