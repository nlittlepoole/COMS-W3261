\documentclass[]{article}
\usepackage{amsmath}
\usepackage{amssymb}
\usepackage{amsthm}
\usepackage{listingsutf8}
\usepackage{multirow}
\usepackage{tikz}
\usepackage{tikz-qtree}
\usepackage{tipa}
\usetikzlibrary{arrows,automata}
\begin{document}

\newtheorem{thm}{Theorem}

\title{COMS W3261 \\ Computer Science Theory \\ Lecture 4\\ Equivalence of
Regular Expressions and Finite Automata}
\author{Alexander Roth}
\date{2014-09-15}
\maketitle
\section*{Outline}
  \begin{itemize}
    \item McNaughton-Yamada-Thompson algorithm: RE to \textepsilon-NFA
    \item Kleene's algorithm: DFA to RE
    \item Converting a DFA to an equivalent RE by eliminating states
    \item A detailed proof that a DFA defines a given language
  \end{itemize}

\section{Review}
  \begin{itemize}
    \item We say that two finite automata are equivalent if they define the same
    language.
    \item In the last two lectures we showed the subset construction could be
    used to convert either an NFA or an \textepsilon-NFA into an equivalent DFA.
    \item We say that a regular expression $E$ and a finite automaton $A$ are
    \emph{equivalent} if $L(E) = L(A)$.
    \item In this lecture we first show how to construct an equivalent
    \textepsilon-NFA from a regular expression. Since we know that we can then
    use the subset construction to convert the \textepsilon-NFA into an
    equivalent DFA, we will then have shown that regular expressions are no more
    powerful in defining languages as a DFA.
    \item We then show that we can construct a regular expression that defines
    the same languages as a DFA.
    \item These results allow us to conclude that DFA's, NFA's,
    \textepsilon-NFA's, and regular expressions are all equivalent in
    definitional power -- they all define precisely the regular languages.
  \end{itemize}
  \subsection*{Class Notes}
    \begin{itemize}
      \item What form of finite automata is this? $\delta: Q \times \Sigma 
      \rightarrow Q$, where $Q \times \Sigma$ is the Domain, and $Q$ is the 
      range. This is a DFA.
      \item $Q = \{q_0, q_1, q_2\}$, $P(Q) = 8 sets$.
      \item If there are only a finite set of state in $Q$, then there is an 
      exponential finite state in the power set of $Q$.
      \item NFA: $\delta: Q\times\Sigma \rightarrow P(Q)$.
      \item $\epsilon$-NFA: $\delta: Q\times(\Sigma\cup\{\epsilon\}) \rightarrow 
      P(Q)$
      \item Regular expressions are an equivalent way of defining the regular 
      languages.
      \item Check out Warthol's algorithm and Floyd's algorithm. They are 
      examples of dynamic programming.
      \item We can prove that a regular expression recognizes the same strings 
      as a finite automata.
    \end{itemize}

\section{McNaughton-Yamada-Thompson Algorithm: From an RE to an equivalent
\textepsilon-NFA}
  \begin{thm}
    Let $R$ be a regular expression $R$. Then we can construct an
    \textepsilon-NFA $N$ such that $L(N) = L(R)$.
  \end{thm}
  \begin{proof}
    See HMU, Sect. 3.2.3, pp. 102 -- 107
  \end{proof}
  \begin{itemize}
    \item The proof is in the form of an algorithm that takes as input a regular
    expression $R$ of length $n$ and recursively constructs from it an
    equivalent \textepsilon-NFA that has
      \begin{itemize}
        \item exactly one start state and one final state,
        \item at most $2n$ states,
        \item no arcs coming into its start state,
        \item no arcs leaving its final state,
        \item at most two arcs leaving any nonfinal state
      \end{itemize}
    \item This algorithm was discovered by McNaughton and Yamada and then
    independently by Ken Thompson who used it in the string-matching program
    \texttt{grep} on Unix. On an input string of length $m$, an $n$-state MYT
    \textepsilon-NFA can be efficiently simulated in time O($mn$) using a
    two-stack algorithm.
  \end{itemize}
  \subsection*{Class Notes}
    Base cases for a regular expression: $a$, $\epsilon$, and $\emptyset$. Here 
    is an example: Let $R = a(a+b)^+$ \\
    \Tree [.concat [.a ][.* [.+ [.a ][.b ]] ]] \\
    \subsubsection*{Two Stack Algorithm}
    If there are $n$ states, in the machine there can be $n$ states in the 
    stack. Sometimes this is called the Two Stack algorithm.

\section{Kleene's Algorithm: From a DFA to an equivalent regular expression}
  \begin{itemize}
    \item Given a DFA $A$, Kleene's algorithm constructs a regular expression
    $R$ from $A$ such that $L(R) = L(A)$.
    \item Suppose the states of $A$ are numbered $1, 2, \ldots, n$.
    \item Kleene's algorithm is a dynamic programming algorithm that constructs
    a regular expressions $R[i, j, k]$ that denotes all paths from state $i$ to
    state $j$ with no intermediate node in the path numbered higher than $k$ as
    follows:

    \newpage
      \begin{lstlisting}
for (i = 1; i <= n; i++)
  for (j= 1; j <= n; j++)
    if (i != j)
      if (there are transitions from state i to
          state j labeled a1, a2, ..., ak)
        R[i,j,0] = a1 + a2 + ... + ak;
      else
        R[i,j,0] = (emptyset);
    else if (i == j)
      if (there are transitions from state i to
          state i labeled a1, a2, ..., ak)
        R[i,i,0] = (emptystring) + a1 + a2 + ... + ak;
      else
        R[i,i,0] = (emptystring)
for (k = 1; k <= n; k++)
  for (i = 1; i <= n; i++)
    for (j = 1; j <= n; j++)
      R[i,j,k] = R[i,j,k-1] + R[i,k,k-1](R[k,k,k-1])*R[k,j,k-1];
      \end{lstlisting}

    \item Assuming the start state is 1, the regular expression for the DFA $A$
    is then the sum (union) of all expressions \texttt{R[1,j,n]} where $j$ is a
    final state.
    \item Note that Kleene's algorithm for constructing a regular expression
    from a DFA reduces to Warshall's transitive closure algorithm and to Floyd's
    all-pairs shortest paths algorithm for directed graphs.
    \item Example 3.5, HMU, pp. 95 -- 97.
  \end{itemize}

\section{Converting a DFA to an equivalent RE by Eliminating States}
  \begin{itemize}
    \item Kleene's algorithm gives us a mechanical way to construct a regular
    expression from a DFA, or for that matter, from any NFA or \textepsilon-NFA.
    \item Another approach that avoids duplicating work is to eliminate states,
    one at a time, from the DFA using the procedure outline in Section 3.2.2 of
    HMU.
    \item Example 3.6, HMU, pp. 101 -- 102.
    \item You can see an expanded treatment of state elimination with more
    examples in pp. 583 -- 588 of Chapter 10 of
    \emph{Aho and Ullman, Foundations of Computer Science}.
  \end{itemize}

\section{A detailed proof that a DFA defines a given language}
  \begin{itemize}
    \item To prove that a DFA $D$ defines a given language $L$, we need to show
    that every string in $L(D)$ is in $L$ and that every string in $L$ is in
    $L(D)$. Here is a detailed example of how this can be done using induction
    for both parts.
    \item Consider the DFA $D = (\{\texttt{A},\texttt{B}\}, \{0, 1\}, \delta,
    \texttt{A}, \{\texttt{A}\})$ in Example 1 from Lecture 2, Sep 8, 2014 where
    the transition function $\delta$ has the transition table: \\

      \begin{tabular}{|c|c|c|}
        \hline
             State & \multicolumn{2}{|c|}{Input Symbol} \\ \cline{2-3}
                   & \texttt{0} & \texttt{1} \\ \hline
        \texttt{A} & \texttt{B} & \texttt{A} \\ \hline
        \texttt{B} & \texttt{A} & \texttt{B} \\ \hline
      \end{tabular}

    \item Let $L$ be the language consisting of all strings of 0's and 1's with
    an even number of zeros. We shall prove that $L(D) = L$.
    \item To begin, we shall show that every string accepted by $D$ is in $L$,
    by proving the following inductive hypothesis by induction on $n$, the
    number of moves made by $D$ accepting a string $w$, for $n \geq 0$: \\
    Inductive Hypothesis IH1:
      \begin{enumerate}
        \item[(a)] If $\delta^n(A, w) = A$, then $w$ has an even numbers of 0's.
        Here, $\delta^n$ means $n$ moves by $D$.
        \item[(b)] If $\delta^n(A, w) = B$, then $w$ has an odd number of 0's.
          \begin{itemize}
            \item Basis. $n = 0$ which implies that $w = \epsilon$ and hence
            that $w$ has an even number of 0's
            \item Induction. Assume IH1 is true for 0, 1, 2, \ldots, $n$ moves.
              \begin{itemize}
                \item Suppose $D$ makes $n + 1$ moves on a string $w = x0$ and
                enters state \texttt{A} after reading $x$ and then enters state
                \texttt{B} after reading the final 0. From the inductive
                hypothesis we know that $x$ must have an even number of 0's.
                Therefore, $x0$ has an odd number of 0's.
                \item Suppose $D$ makes $n + 1$ moves on a string $w = x0$ and
                enters state \texttt{B} after reading $x$ and then enters state
                \texttt{A} after reading the final 0. From the inductive
                hypothesis we know that $x$ must have an odd number of 0's.
                Therefore, $x0$ has an even number of 0's.
                \item Suppose $D$ makes $n + 1$ moves on a string $w = x1$ and
                enters state \texttt{A} after reading $x$ and then state
                \texttt{A} after reading the final 1. From the inductive
                hypothesis we know that $x$ must have an even number of 0's.
                Therefore, $x1$ also has an even number of 0's.
                \item Suppose $D$ makes $n + 1$ moves on a string $w = x1$ and
                enters state \texttt{B} after reading $x$ and then state
                \texttt{A} after reading the final 1. From the inductive
                hypothesis we know that $x$ must have an odd number of 0's.
                Therefore, $x1$ also has an odd number of 0's.
              \end{itemize}
          \end{itemize}
      \end{enumerate}
    \item We have now shown that IH1 is true for all sequences of $n$ moves,
    where $n \geq 0$.
    \item We now need to show that every string in $L$ is accepted by $D$.
    To do this, we shall prove the following inductive hypothesis by
    induction on $n$, the length of a string $w$, for $n \geq 0$: \\
    Inductive Hypothesis IH2:
      \begin{enumerate}
       \item[(a)] If $w$ has an even number of 0's, then
        $\delta^*(A,w) = A$.
        \item[(b)] If $w$ has an odd number of 0's, then
        $\delta^*(A, w) = B$
          \begin{itemize}
            \item Suppose $w$ is now a string of length $n + 1$ and $w = x0$
            and $x$ has an even number of 0's. From IH2,
            $\delta^*(A, x) = A$. Since $\delta(A, 0) = B$,
            $\delta^*(A, w) = B$.
            \item Suppose $w = x0$ and $x$ has an odd number of 0's. From
            IH2, $\delta^*(A, x) = B$. Since $\delta(B, 0) = A$,
            $\delta^*(A, w) = A$.
            \item Suppose $w = x1$ and $x$ has an even number of 0's. From
            IH2, $\delta^*(A, x) = A$. Since $\delta(A, 1) = A$,
            $\delta^*(A, w) = A$.
            \item Suppose $w = x1$ and $x$ has an odd number of 0's. From
            IH2, $\delta^*(A, x) = B$. Since $\delta(B, 1) = B$,
            $\delta^*(A, w) = B$.
          \end{itemize}
      \end{enumerate}
    \item We have now shown that IH2 is true for all strings of length $n$,
    $n \geq 0$.
    \item From the two inductive hypotheses, we can conclude that $w$ is
    accepted by $D$ iff $w$ is in $L$. In other words, $L(D) = L$.
  \end{itemize}

\section{Practice Problems}
  \begin{enumerate}
    \item Use the MYT algorithm to construct an equivalent \textepsilon-NFA for
    the regular expression $a + b * a$.
    \item Show the behavior of your \textepsilon-NFA on the input string
    \texttt{bba}.
    \item Use the subset construction to convert your \textepsilon-NFA into a
    DFA.
    \item Show the behavior of your DFA on the input string \texttt{bba}.
    \item Consider the DFA $D$ with:
      \begin{enumerate}
        \item[1.] $Q = \{\texttt{1, 2, 3}\}$
        \item[2.] $\Sigma = \{\texttt{a, b}\}$
        \item[3.] $\delta$: \\

          \begin{tabular}{|c|c|c|}
            \hline
            State      & \multicolumn{2}{|c|}{Input Symbol} \\ \cline{2-3}
                       & \texttt{a} & \texttt{b}            \\ \hline
            \texttt{1} & \texttt{2} & \texttt{1}            \\ \hline
            \texttt{2} & \texttt{3} & \texttt{1}            \\ \hline
            \texttt{3} & \texttt{3} & \texttt{2}            \\ \hline
          \end{tabular}

        \item[4.] Start state: \texttt{1}
        \item[5.] $F = \{\texttt{3}\}$
          \begin{enumerate}
            \item[a)] Use Kleene's algorithm to construct a regular expression
            for $L(D)$. Simplify your expressions as much as possible at each
            stage.
            \item[b)] Construct a regular expression for $L(D)$ by eliminating
            state 2.
          \end{enumerate}
      \end{enumerate}
  \end{enumerate}

\section{Reading Assignment}
  \begin{itemize}
    \item HMU: Ch 3
  \end{itemize}

\end{document}
