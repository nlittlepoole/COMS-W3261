\documentclass[]{article}
\usepackage{amsmath}
\usepackage{amssymb}
\usepackage{amsthm}
\usepackage{listings}
\usepackage{multirow}
\usepackage{tikz}
\usepackage{tikz-qtree}
\usepackage{tipa}
\usetikzlibrary{arrows,automata}
\begin{document}
\newcommand*{\xml}[1]{\texttt{<#1>}}
\theoremstyle{definition}
\newtheorem{thm}{Theorem}

\title{COMS W3261 \\ Computer Science Theory \\ Lecture 12\\ Turing Machines}
\author{Alexander Roth}
\date{2014 -- 10 -- 13}
\maketitle

\section*{Outline}
  \begin{enumerate}
    \item Turing machines
    \item Algorithms and recursive languages
    \item Some history
    \item Models of computation equivalent to Turing machines
  \end{enumerate}
  
\section{Turing Machines}
  \begin{itemize}
    \item A Turing machine is a generalization of a finite automaton. At any
    moment in time, it can be pictured as a finite-state control with a tape 
    head reading a symbol on a square of its infinite input tape. Initially, a
    finite length input string $a_1a_2\ldots{a_n}$ appears on the input tape
    with the tape head reading $a_1$ and the finite control in a designated
    initial state. The symbols of the input string appear in contiguous squares
    of the input tape. \\
    An infinite number of blank symbols are on the input tape to the left of 
    $a_1$ and to the right of $a_n$.
    \item At the beginning of a move, a Turing machine reads the symbol on the 
    square of the input tape under the tape head and consults the transition 
    function (its ``program'') stored in its finite-state control. During the 
    move it makes a state transition, replaces the symbol on the input tape 
    with another tape symbol, and shifts the tape head one square to the left 
    or one square to the right. If it has not entered a halting state, the 
    Turing machine then makes another move.
    \item After a finite (but perhaps very long) number of moves the Turing 
    machine may enter a final state and halt, in which case it is said to 
    accept the input string $a_1a_2\ldots{}a_n$ that was originally on the 
    input tape. However, the Turing machine may instead enter a nonfinal state 
    and halt, or it may make an infinite sequence of moves without ever 
    entering a final state. In both cases, it does not accept the original 
    input string.
    \item More formally, a nondeterministic Turing machine $M$ has seven 
    components: $(Q, \Sigma, \Gamma, \delta, q_0, B, F)$
      \begin{enumerate}
        \item $Q$ is the finite set of states of the finite control.
        \item $\Sigma$ is the finite set of input symbols.
        \item $\Gamma$ is the finite set of tape symbols; $\Sigma$ is a subset 
        of $\Gamma$.
        \item $\delta$ is the transition function. It maps $(Q \times \Gamma)$ 
        to the subsets of $(Q \times \Gamma \times \{L,R\})$. If $(p, Y, D)$ is 
        in $\delta(q,X)$ and $M$ is in state $q$ reading the symbol $X$ on the 
        input tape, then $M$ can
          \begin{itemize}
            \item go from state $q$ to state $p$,
            \item replace the symbol $X$ on the input tape by the symbol $Y$, 
            and
            \item move its input head one square in the direction $D$ where $D$ 
            can be either $L$ (for left) or $R$ (for right).
          \end{itemize}
        M is \emph{deterministic} if there is at most one element in $
        \delta(q,X)$ for any state $q$ and tape symbol $X$. Unless otherwise 
        qualified, the term ``Turing machine'' will signify a deterministic 
        Turing machine.
        \item $q_0$ is the start state.
        \item $B$ is the blank symbol. $B$ is in $\Gamma$ but not in $\Sigma$.
        \item $F$, a subset of $Q$, is the set of final accepting states. We 
        assume there are no transitions from a final state so that when $M$ 
        enters a final state it halts.
      \end{enumerate}
    \item During a sequence of moves we can represent the configuration of $M$ 
    by an instantaneous description (ID) of the form $X_1X_2\ldots{}X_{i-1}
    qX_iX_{i+1}\ldots{}X_n$. This ID says $M$ is in state $q$ reading the tape 
    symbol $X_i$. To the left of $X_i$ is the string of tape symbols 
    $X_1X_2\ldots{}X_{i-1}$. To the right of $X_i$ is the string of tape 
    symbols $X_{i+1}\ldots{}X_n$. We do not show blanks on the input tape 
    unless necessary.
    \item The read head may be one input square past the right end of the input 
    reading a blank in which case $M$ is in the ID
      \[ X_1X_2\ldots{}X_nqB \]
    \item Moves by $M$ are modeled by the following changes to the ID:
      \begin{enumerate}
        \item If $(p, Y, L)$ is in $\delta(q, X_i)$, $M$ can move to the ID
          \[ X_1X_2\ldots{}X_{i-2}pX_{i-1}YX_{i+1}\ldots{}X_n \]
        If $i = 1$, then $M$ moves to the blank to the left of $X_1$ in which 
        case the ID becomes 
          \[ pBYX_2\ldots{}X_n \]
        If $i = n$ and $Y = B$, then $M$ replaces $X_n$ by a blank and the ID 
        becomes
          \[ X_1X_2\ldots{}X_{n-2}pX_{n-1} \]
        \item If $(p, Y, R)$ is in $\delta(q, X_i)$, $M$ can move to the ID
          \[ X_1X_2\ldots{}X_{i - 1}YpX_{i + 1}\ldots{}X_n \]
        If $i = n$, then $M$ moves to the blank to the right of $X_n$ in which
        case the ID becomes
          \[ X_1X_2\ldots{}X_{n - 1}YpB \]
        If $i = 1$ and $Y = B$, then $M$ replaces $X_1$ by a blank and the ID 
        becomes
          \[ pX_2\ldots{}X_n \]
      \end{enumerate}
    \item $L(M)$, the language accepted by $M$, is the set of strings $w$ in 
    $\Sigma^*$ such that $q_0w\overset{*}{\vdash} \alpha{p}\beta$ for some 
    state $p$ in $F$.
    \item We say that a language $L$ is \emph{recursively enumerable} if 
    $L = L(M)$ for some Turing machine $M$.
    \item See Example 8.2, p. 329, HMU for a TM that accepts the language
    $\{0^n1^n \, | \, n \geq 1 \}$.
  \end{itemize}

\section{Algorithms and Recursive Languages}
  \begin{itemize}
    \item We say that a language $L$ is \emph{recursive} if $L = L(M)$ for some 
    Turing machine $M$ such that
      \begin{enumerate}
        \item If $w$ is in $L$, then $M$ accepts $w$ and therefore halts.
        \item If $w$ is not in $L$, then $M$ eventually halts but never enters 
        an accepting state.
      \end{enumerate}
    \item A Turing machine that hatls on all inputs either in an accepting or 
    nonaccepting state provides a precise definition for the term 
    \emph{algorihtm}. This formalizes our intuitive notion of an algorithm as a 
    collection of simple instructions for carrying out some task.
    \item A language $L$ is said to be \emph{decidable} if it is a recursive 
    language.
    \item A language $L$ is said to be \emph{undecidable} if it is not a 
    recursive language.
  \end{itemize}
  \subsection*{Class Notes}
    \begin{enumerate}
      \item Given a CFG $G$, is $G$ ambiguous?
      \item Given a CFG $G$, is there an equivalent unambiguous grammar?
    \end{enumerate}

\section{Some History}
  \begin{itemize}
    \item Hilbert's problems.
      \begin{itemize}
        \item In 1900, David Hilbert, the eminent mathematician of his day, 
        presented a famous list of twenty-three problems at the International 
        Congress of Mathematicians in Paris.
        \item Hilbert's tenth problem was to devise a process according to 
        which it can be determined by a finite number of operations whether a 
        polynomial with integer coefficients (Diophantine equation) has an 
        integral root.
          \begin{itemize}
            \item For example, the polynomial Diophantine equation
              \[ 6x^3yz^2 + 3xy^2 - x^3 - 10 = 0 \]
            over the variables $x$, $y$, and $z$ has an integral root at 
            $x = 5$, $y = 3$, and $z = 0$. The root is said to be integral 
            because all variables are assigned integer values.
          \end{itemize}
      \end{itemize}
    \item In 1931 Kurt G\"odel published his famous incompleteness theorems
    which basically said that in any reasonable system of formalizing the 
    notion of provability in number theory, some true statements are 
    unprovable. This shattered Hilbert's dream of finding a complete and 
    consistent set of axioms for all of mathematics (Hilbert's second problem).
    \item In 1936 Alonzo Church wrote a paper in which he devised a notation
    called lambda calculus to define algorithms. Lambda calculus is the basis
    for the programming language Lisp.
    \item In 1936 Alan turing wrote a paper (``On computable numbers with an 
    application to the Entsheidungsproblem'', Proc. London Math. Society) in 
    which he defined Turing machines and showed that the halting problem for 
    Turing machines is undecidable.
      \begin{itemize}
        \item The machine $M$ in Alan Turing's paper accepted by just halting 
        -- there is no final state. Let $H(M)$ be the set of inputs $w$ on 
        which his machine hatls. Turing showed that set of pairs $(M, w)$ such 
        that $w$ is in $H(M)$ is recursively enumerable but not recursive.
      \end{itemize}
    \item The Church-Turing thesis hypothesizes that any function that can be 
    computed (``effectively computable function'') can be computed on a Turing 
    machine.
    \item In 1970, Yuri Matiyasevich showed that no algorithm exists for 
    testing whether a Diophantine equation has integral roots.
  \end{itemize}

\section{Models of Computation Equivalent to Turing Machines}
  \begin{itemize}
    \item Many variants of Turing machines have been defined such as:
      \begin{itemize}
        \item Turing machines with a semi-infinite input tape.
        \item Multitape Turing machines.
        \item Turing machines with tapes having multiple tracks.
        \item Nondeterministic Turing machines.
      \end{itemize}
    \item All these machines are equivalent to our definition of a 
    deterministic Turing machines.
    \item Other universal models of computation:
      \begin{itemize}
        \item Chomsky type 0 grammars. A type 0 grammar is like a context-free 
        grammar $(V, T, P, S)$ except that productions can be of the form
        $\alpha \rightarrow \beta$ where $\alpha$ is a string of nonterminals 
        and terminals with at least one nonterminal and $\beta$ is any string 
        of nonterminals and nonterminals.
        \item Lambda calculus.
        \item Pushdown automata with two or more stacks.
        \item Two-counter machines.
        \item Random access machines.
        \item Most programming languages.
        \item Real computers with an arbitrary amount of energy.
      \end{itemize}
    \item Again, all these models are computationally equivalent to our 
    definition of a deterministic Turing machine.
  \end{itemize}
\end{document}