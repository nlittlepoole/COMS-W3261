\documentclass[]{article}
\usepackage{amsmath}
\usepackage{amssymb}
\usepackage{amsthm}
\usepackage{listings}
\usepackage{multirow}
\usepackage{tikz}
\usepackage{tikz-qtree}
\usepackage{tipa}
\usetikzlibrary{arrows,automata}
\begin{document}
\newcommand*{\xml}[1]{\texttt{<#1>}}
\theoremstyle{definition}
\newtheorem{thm}{Theorem}
\title{COMS W3261 \\ Computer Science Theory \\ Lecture 19 \\ Satisfiability}
\author{Alexander Roth}
\date{2014 -- 11 -- 12}

\maketitle

\section*{Outline}
\begin{itemize}
\item Boolean expressions
\item The satisfiability problem
\item Normal forms for boolean expressions
\item The problems CSAT and kSAT
\item SAT is NP-complete: The Cook-Levin theorem
\end{itemize}

\section{Boolean Expressions}
\begin{itemize}
\item Boolean expressions are generated by the following CFG:
\begin{align*}
E &\rightarrow E \vee   T \,|\,T \\
T &\rightarrow T \wedge F \,|\,F \\
F &\rightarrow ( E )      \,|\,\neg\,|\,var
\end{align*}
\emph{var} represents a variable whose value can be either 1 (for true) or 0
(for false). The operator $\wedge$ stands for logical AND, $\vee$ for logical
OR, and $\neg$ for logical NOT. Note that the grammar gives the operator $\neg$
the highest precedence, then $\wedge$, then $\vee$.
\end{itemize}
\subsection*{Class Notes}
\subsubsection*{Normal Forms for BE's}
\begin{itemize}
\item Conjunction: AND $x \wedge y$ xy
\item Disjunction: OR $x \vee y$ $x + y$
\item Negation: NOT $\neg x$ $\overline{x}$
\item Literal: variable or negated variable $x$, $\overline{x}$
\item Clause: logical OR of one or more literals.
\end{itemize}
A BE is in Conjunctive Normal Form (CNF) if it is the logical AND of clauses,
e.g., $(x + \overline{y})(\overline{x} + y) = (x \vee \neg y)\wedge(\neg x \vee
y)$

\section{The Satisfiability Problem}
\begin{itemize}
\item A truth assignment for a boolean expression assigns either the value true
(1) or the value false(0) to each of the variables in the expression.
\item The value $E(T)$ of an expression $E$ given a truth assignment $T$ is the
result of evaluating $E$ with each variable $x$ in $E$ replaced by $T(x)$.
\item A truth assignment $T$ satisfies $E$ if $E(T) = 1$.
\item An expression $E$ is \emph{satisfiable} if there exists a truth assignment
$T$ that satisfies $E$.
\item The \emph{satisfiability problem} (SAT) is to determine whether a given
boolean expression is satisfiable.
\begin{itemize}
\item The value of $E = x \wedge \neg(y \vee z)$ given the truth assignment
$T(x) = 1$, $T(y) = 0$, $T(z) = 0$ is $1$. $\lbrack 1 \wedge \neq(0 \vee 0) = 1
\rbrack$. Thus, $E$ is satisfiable.
\item The expression $E = x \wedge(\neq x \vee y ) \wedge \neg y$ is not
satisfiable because none of the four truth assignments to the variables $x$ and
$y$ causes $E$ to have the value 1.
\end{itemize}
\item We shall shortly prove that SAT is NP-complete.
\end{itemize}
\subsection*{Class Notes}
SAT
Given a BE $E$, is $E$ satisfiable?
\begin{thm}
(Cook-Levin Theorem) SAT is NP-complete
\end{thm}
\begin{enumerate}
\item SAT is in NP.
\item Every problem in NP can be reduced in polynomial time to SAT.
\end{enumerate}

\section{Normal Forms for Boolean Expressions}
\begin{itemize}
\item In boolean expressions
\begin{itemize}
\item Logical AND, as in $x \wedge y$, is often called conjunction and is
sometimes written as a product, as in $xy$.
\item Logical OR, as in $x \vee y$, is often called disjunction and is sometimes
written as a sum, as in $x + y$.
\item Logical NOT, as in $\neg x$, is often called negation and is sometimes
written with an overbar, as in $\overline{x}$.
\item A literal is a variable or a negated variable; e.g., $x$ and $\neg x$ are
both literals.
\item A clause is the logical OR (disjunction) of one or more literals; e.g., $x
\vee \neg y$ is a clause.
\end{itemize}
\item A boolean expression is in conjunctive normal form (CNF) if it is the
local AND (conjunction) of clauses; e.g., $(x \vee \neg y) \wedge (\neg x \vee
z)$ is in CNF.
\item A boolean expression is in $k$-CNF if it is the logical AND of clauses
each one of which is the logical OR of exactly $k$ distinct literals; e.g., $(w
\vee \neg x \vee y) \wedge (x \vee \neg y \vee z)$ is in 3-CNF.
\item Two boolean expressions are equivalent if they have the same result on any
truth assignment to their variables.
\end{itemize}

\section{The Problems CSAT and kSAT}
\begin{itemize}
\item CSAT
\begin{itemize}
\item Given a boolean expression $E$ in $k$-CNF, is $E$ satisfiable?
\item We can view CNF as the language $\{ E\,|\,E$ is the representation of a
satisfiable CNF boolean expressions $\}$.
\item CST is NP-complete.
\end{itemize}
\item kSAT
\begin{itemize}
\item Given a boolean expression $E$ in $k$-CNF, is $E$ satisfiable?
\item 1SAT and 2SAT are in P; kSAT is NP=complete for $k \geq 3$.
\end{itemize}
\end{itemize}

\section{SAT is NP-complete: the Cook-Levin theorem}
\begin{itemize}
\item SAT is in NP
\begin{itemize}
\item Given a boolean expression $E$ of length $n$, a multitape nondeterministic
Turing machine can guess a truth assignment $T$ for $E$ in $O(n)$ time.
\item The NTM can then evaluate $E$ using the truth assignment $T$ in $O(n^2)$
time.
\item If $E(T) = 1$, then the NTM accepts $E$.
\item The NTM can be simulated by a single-tape deterministic TM in $O(n^4)$
time.
\end{itemize}
\item If $L$ is in NP, then there is a polynomial-time reduction of $L$ to SAT.
\item If a NTM $M$ accepts an input $w$ of length $n$ in $p(n)$ time, then $M$
has a sequence of moves such that
\begin{enumerate}
\item $\alpha_0$ is the initial ID of $M$ with input $w$.
\item $\alpha_0 \vdash \alpha_1 \vdash \ldots \vdash \alpha_k$ where $k \leq
p(n)$.
\item $\alpha_k$ is an ID with an accepting state.
\item Each $\alpha_i$ consists only of nonblanks unless $\alpha_i$ ends in a
state and a blank, and extends from the initial head position to the right.
\end{enumerate}
\item From $M$ and $w$, we can construct a boolean expression $E_{M,w}$ that is
satisfiable iff $M$ accepts $w$ within $p(n)$ moves. See HMU, pp. 400--446 for
details.
\end{itemize}
\subsection*{Class Notes}
\[E_{M,w} = U \wedge S \wedge S \wedge N \wedge F \]
where
\begin{itemize}
\item $U$ is the unique symbol in each cell
\item $S$ starts off right
\item $N$ next move is right
\item $F$ finishes right.
\end{itemize}

\section*{Class Notes}
\subsection*{Some Examples of NP-Complete Problems}
\begin{itemize}
\item SAT
\item CSAT
\end{itemize}
\end{document}
