\documentclass[]{article}
\usepackage[all]{xy}
\usepackage{amsmath}
\usepackage{amssymb}
\usepackage{amsthm}
\usepackage{enumitem}
\usepackage{indentfirst}
\usepackage{listings}
\usepackage{multirow}
\usepackage{tikz}
\usepackage{tikz-qtree}
\usepackage{tipa}
\begin{document}

\newtheorem{thm}{Theorem}
\title{Computer Science Theory \\ COMS W3261 \\ Lecture 22}
\author{Alexander Roth}
\date{2014 -- 11 -- 24}
\maketitle

\section*{Outline}
\begin{enumerate}
\item Normal form
\item Reduction strategies
\item The Church-Rosser theorems
\item The Y combinator
\item Implementing factorial using the Y combinator
\end{enumerate}

\section{Normal Form}
\begin{itemize}
\item An expression containing no possible beta reductions is said to be in
normal form. A normal form expression is one containing no redexes (reducible
expressions), that is, one with no subexpressions of the form $(\lambda
x.f\,g)$.
\item Examples of normal form expressions:
\begin{itemize}
\item $x$ where $x$ is a variable.
\item $x\,e$ where $x$ is a variable and $e$ is a normal form expression.
\item $\lambda x.e$ where $x$ is a variable and $e$ is a normal form expression.
\end{itemize}
\item The expression $(\lambda x.x\,x)(\lambda x.x\,x)$ does not have a normal
form because it is a redex that always evaluates to itself. We can think of this
expression as a representation for an infinite loop.
\end{itemize}

\section{Reduction Strategies}
\begin{itemize}
\item A reduction strategy specifies the order in which beta reductions for a
lambda expression are made.
\item Some reduction orders for a lambda expression may yield a normal form
while other orders may not. For example, consider the given expression
\[ (\lambda x.1)((\lambda x.x\,x)(\lambda x.x\,x)) \]
This expression has two redexes:
\begin{enumerate}
\item The entire expression is a redex in which we can apply the function
$(\lambda x.1)$ to the argument $((\lambda x.x\,x)(\lambda x.x\,x))$ to yield
the normal form 1. This redex is the leftmost outermost redex in the given
expression.
\item The subexpression $((\lambda x.x\,x)(\lambda x.x\,x))$ is also a redex in
which we can apply the function $(\lambda x.x\,x)$ to the argument $(\lambda
x.x\,x)$. Note that this redex is the leftmost innermost redex in the given
expression. But if we evaluate this redex we get same subexpression: $(\lambda
x.x\,x)(\lambda x.x\,x)\rightarrow(\lambda x.x\,x)(\lambda x.x\,x)$. Thus,
continuing to evaluate the leftmost innermost redex will not terminate and no
normal form will result.
\end{enumerate}
\item As a second example, consider the expression
\[ (\lambda x.\lambda y.y)((\lambda z.z\,z)(\lambda z.z\,z)) \]
This expression has two redexes:
\begin{enumerate}
\item The entire expression is a redex in which we apply the function $(\lambda
x.\lambda y.y)$ to the argument $((\lambda z.z\,z)(\lambda z.z\,z))$ to yield
the normal form $(\lambda y.y)$. This redex is the leftmost outermost redex in
the given expression.
\item The subexpression $((\lambda z.z\,z)(\lambda z.z\,z))$ is also a redex in
which we apply the function $(\lambda z.z\,z)$ to the argument $(\lambda
z.z\,z)$. Note that this redex is the leftmost innermost redex in the given
expression. But if we evaluate this redex we get the same subexpression:
$((\lambda z.z\,z)(\lambda z.z\,z))\rightarrow((\lambda z.z\,z)(\lambda z.z))$.
Thus, continuing to evaulate the leftmost innermost redex will not terminate and
no normal will result.
\end{enumerate}
\item There are two common reduction orders for lambda expression: normal order
evaluation and applicative order evaluation
\begin{description}
\item[Normal order evaluation]:
\begin{itemize}
\item In normal order evaluation we always reduce the leftmost outermost redex
at each step.
\item The first reduction order in each of the two examples above is a normal
order evaluation.
\item A remarkable property of lambda calculus is that every lambda expression
has a unique normal form if one exists. Moreover, if an expression has a normal
form, then normal order evaluation will always find it.
\end{itemize}
\item[Applicative order evaluation]:
\begin{itemize}
\item In applicative order evaluation we always reduce the leftmost innermost
redex at each step.
\item Applicative order evaluates the arguments of a function before evaluating
the function itself.
\item The second reduction order in each of the two examples above isa n
applicative order evaluation.
\item Thus, even though an expression may have a normal form, applicative order
evaluation may fail to find it.
\end{itemize}
\end{description}
\end{itemize}

\section{The Church-Rosser Theorems}
\begin{itemize}
\item A remarkable property of lambda calculus is that every expression has a
unique normal form if one exists.
\item \textbf{Church-Rosser Theorem I:} If $e\overset{*}{\rightarrow}f$ and
$e\overset{*}{\rightarrow}g$ by any two reduction orders, then there always
exists a lambda expression $h$ such that $f\overset{*}{\rightarrow}h$ and
$g\overset{*}{\rightarrow}h$.
\begin{itemize}
\item A corollary of this theorem is that no lambda expression can be reduced to
two distinct normal forms. To see this, suppose $f$ and $g$ are in normal form.
The Church-Rosser theorem says there must be an expression $h$ such that $f$ and
$g$ are each reducible to $h$. Since $f$ and $g$ are in normal form, they cannot
have any redexes so $f = g = h$.
\item This corollary says that all reduction sequences that terminate will
always yield the same result and that result must be a normal form.
\item The term \emph{confluent} is often applied to a rewriting system that has
the Church-Rosser property.
\end{itemize}
\item \textbf{Church-Rosser Theorem II:} If $e\overset{*}{\rightarrow}f$ and $f$
is in normal form, then there exists a normal order reduction sequence from $e$
to $f$.
\end{itemize}

\section{The Y Combinator}
\begin{itemize}
\item The $Y$ combinator (sometimes called the paradoxical combinator) is a function that takes a function $G$ as an argument and returns $G(YG)$. With repeated applications we can get $G(G(YG)), G(G(G(YG))),\ldots$.
\item We can implement recursive functions using the $Y$ combinator.
\item $Y$ is defined as follows:
\[(\lambda f.(\lambda x.f(x\,x))(\lambda x.f(x\,x)))\]
\item Let us evaluate $YG$ where $G$ is an expression:
\begin{align*}
(\lambda f.(\lambda x.f(x\,x))(\lambda x^\prime.f(x\prime\,x^\prime)))G &\rightarrow(\lambda x.G(x\,x))(\lambda x^\prime.G(x^\prime\,x^\prime)) \\
& \rightarrow G((\lambda x^\prime.G(x^\prime\,x^\prime))(\lambda x^\prime.G(x^\prime\,x^\prime))) \\
& \leftrightarrow G((\lambda f.(\lambda x.f(x\,x))(\lambda x.f(x\,x)))G) \\
& = G(YG)
\end{align*}
\item Thus, $YG \overset{*}{\rightarrow} G(YG)$; that is, $YG$ reduces to a call of $G$ on $(YG)$.
\item We will use $Y$ to implement recursive functions.
\item $Y$ is an example of a fixed-point combinator.
\end{itemize}

\section{Implementing Factorial using the Y Combinator}
\begin{itemize}
\item If we could name lambda abstractions, we could define the factorial function with the following recursive definition:
\[ FAC = (\lambda n.IF\,(=\,n\,0)\,1\,(*\,n\,(FAC\,(-\,n\,1)))) \]
where $IF$ is a conditional function.
\item However, functions in lambda calculus cannot be named; they are anonymous.
\item But we can express recursion as the fixed-point of a function $G$. To do this, let us simplify the essence of the problem. We begin with a skeletal recursive definition:
\[ FAC = \lambda n.(\ldots FAC \ldots) \]
\item By performing the beta abstraction on $FAC$, we can transform its definition to:
\begin{align*}
FAC &= (\lambda f.(\lambda n.(\ldots f \ldots)))\, FAC \\
    &= G\,FAC
\end{align*}
where
\[ G = \lambda f.\lambda n.IF\,(=\,n\,0)\,1\,(*\,n\,(f\,(-\,n\,1))) \]
Beta abstraction is just the reverse of beta reduction.
\item The equation
\[ FAC = G\,FAC \]
says that when the function $G$ is applied to $FAC$, the result is $FAC$. That is, $FAC$ is a fixed-point of $G$.
\item We can use the $Y$ combinator to implement $FAC:$
\[ FAC = Y\,G \]
\item As an example, let's compute $FAC\,1$:
\begin{align*}
FAC\,1 &= Y\,G\,1      \\
&= G\,(Y\,G)\,1 \\
&= \lambda f.\lambda n.IF\,(=\,n\,0)\,1\,(*\,n\,(f\,(-\,n\,1)))(Y\,G)\,1 \\
&\rightarrow \lambda n.IF\,(=\,n\,0)\,1\,(*\,n\,((Y\,G)(-\,n\,1)))\,1    \\
&\rightarrow IF (=\,n\,0)\,1\,(*\,n\,((Y\,G)(-\,1\,1))) \\
&\rightarrow *\,1\,(Y\,G\,0) \\
&= *\,1\,(G(Y\,G)\,0) \\
&= *\,1((\lambda f.\lambda n.IF\,(=\,n\,0)\,1\,(*\,n\,(f\,(-\,n\,1))))(Y\,G)\,0)\\
&\rightarrow *\,1((\lambda n.IF\,(=\,n\,0)\,1\,(*\,n\,((Y\,G)(-\,n\,1)))))0 \\
&\rightarrow *\,1(IF\,(=\,0\,0)\,1\,(*\,0\,((Y\,G)(-\,0\,1)))) \\
&\rightarrow *\,1\,1 \\
&\rightarrow 1     
\end{align*}
\end{itemize}

\section*{Class Notes}
\subsection*{Beta Reduction}
\begin{align*}
&(\lambda x.\lambda y.+\,x\,y)\,1\,2 \\
&(+\,x\,y) \\
&(\lambda y.(\lambda y.(+\,x\,y))) \\
&((\lambda x.(\lambda y.(+\,x\,y)))1)2 \\
&(\lambda y.(+\,1\,y)) 2 \\
&(+\,1\,2)
\end{align*}

\subsection*{Argument is a Function}
\begin{align*}
&(\lambda f.f\,1)(\lambda x.(+\,x\,2)) \\
&\overset{\beta}{\rightarrow} (\lambda x.(+\,x\,2)) 1\\
&\overset{\beta}{\rightarrow} (+\,1\,2)
\end{align*}

\subsection*{Rename to avoid name conflicts}
\begin{align*}
&((\lambda x.(\lambda y.(x\,y))) y) \\
&\overset{\alpha}{\rightarrow} ((\lambda x.(\lambda z.(x\,z))) y) \\
&\overset{\beta}{\rightarrow} (\lambda z.(y\,z))
\end{align*}

\section*{The Lambda Calculus II}
\begin{description}
\item[Normal form:] A lambda expression with no redexes
\end{description}
\subsubsection*{Examples}
\begin{itemize}
\item $x$
\item $\lambda y$
\item $\lambda x.y$
\end{itemize}

Consider 
\begin{align*}
&(\lambda x.x\,x)(\lambda x.x\,x)\\
&\overset{\beta}{\rightarrow}(\lambda x.x\,x)(\lambda x.x\,x)
\end{align*}
Infinite loop.

\subsection*{Reduction Strategy}
The order in which beta reductions are made.
\[ e = ((\lambda x.f)g) \]
$eh$: The lambda in $e$ is to the left of any lambda in $h$, $f$, $g$.

\end{document}
