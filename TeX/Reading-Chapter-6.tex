\documentclass[]{article}
\usepackage{amsmath}
\usepackage{amssymb}
\usepackage{amsthm}
\usepackage[T1]{fontenc}
\usepackage{indentfirst}
\usepackage{tikz}
\usepackage{tikz-qtree}
\usepackage{tipa}
\usetikzlibrary{arrows,automata}
\begin{document}

\title{COMS W3261 \\ Computer Science Theory \\ Chapter 6 Notes}
\author{Alexander Roth}
\date{2014--09--27}
\maketitle
\theoremstyle{definition}
\newtheorem{thm}{Theorem}

\section*{Pushdown Automata}
  The context-free languages have a type of automaton that defines them. This
  automaton, called a ``pushdown automaton,'' is an extension of the
  nondeterministic finite automaton with $\epsilon$-transitions, which is one of
  the ways to define the regular languages. It is essentially an $\epsilon$-NFA
  with the addition of a stack. The stack can be read, pushed, and popped only
  at the top, just like the ``stack'' data structure.

\section*{Definition of the Pushdown Automaton}
  \subsection*{Informal Induction}
    The pushdown automaton is in essence a nondeterministic finite automaton
    with $\epsilon$-transitions permitted and one additional capability: a stack
    on which it can store a string of ``stack symbols''. This stack allows the
    automaton to ``remember'' an infinite amount of information. \\
    \indent Push-down automata recognize all and only the context-free
    languages. \\
    \indent A ``finite-state control'' reads inputs, one symbol at a time. The
    pushdown automaton is allowed to observer the symbol at the top of the stack
    and to base its transition on its current state, the input symbol, and the
    symbol at the top of stack. Alternatively, it may make a ``spontaneous''
    transition, using $\epsilon$ as its input instead of an input symbol. In one
    transition, the pushdown automaton:
      \begin{enumerate}
        \item Consumes from the input the symbol that it uses in the transition.
        If $\epsilon$ is used for the input, then no input symbol is consumed.
        \item Goes to a new state, which may or may not be the same as the
        previous state.
        \item Replaces the symbol at the top of the stack by any string. The
        string could be $\epsilon$, which corresponds to a pop of the stack. It
        could be the same symbol that appeared at the top of the stack
        previously; i.e., no change to the stack is made. It could also replace
        the top stack symbol by one other symbol, which in effect changes the
        top of the stack but does not push or pop it. Finally, the top stack
        symbol could be replaced by two or more symbols, which has the effect of
        (possibly) changing the top stack symbol, and then pushing one or more
        new symbols onto the stack.
      \end{enumerate}

  \subsection*{The Formal Definition of Pushdown Automata}
    Our formal notation for a \emph{pushdown automaton} (PDA) involves seven
    components. We write the specification of a PDA $P$ as follows:
      \[ P = (Q,\Sigma,\Gamma,\delta,q_0,Z_0,F) \]
    The components have the following meanings:
      \begin{description}
        \item[$Q$:] A finite set of \emph{states}, like the states of a finite
        automaton.
        \item[$\Sigma$:] A finite set of \emph{input symbols}, also analogous to
        the corresponding components of a finite automaton.
        \item[$\Gamma$:] A finite \emph{stack alpahbet}. This component, which
        has no finite-automaton analog, is the set of symbols that we are
        allowed to push onto the stack.
        \item[$\delta$:] The \emph{transition function}. As for a finite
        automaton, $\delta$ governs the behavior of the automaton. Formally, $
        \delta$ takes as arguments a triple $\delta(q,aX)$, where:
          \begin{enumerate}
            \item $q$ is a state in $Q$.
            \item $a$ is either an input symbol in $\Sigma$ or $a = \epsilon$,
            the empty string, which is assumed not to be an input symbol.
            \item $X$ is a stack symbol, that is, a member of $\Gamma$.
          \end{enumerate}
        The output of $\delta$ is a finite set of pairs $(p,\gamma)$, where $p$
        is the new state, and $\gamma$ is the string of stack symbols that
        replace $X$ at the top of the stack.
        \item[$q_0$:] The \emph{start state}. The PDA is in this state before
        making any transitions.
        \item[$F$:] The set of \emph{accepting states}, or \emph{final states}.
      \end{description}

    \subsubsection*{No ``Mixing and Matching''}
      There may be several pairs that are options for a PDA in some situation.
      for instance suppose $\delta(q,aX) = \{(p,Y,Z),(r,\epsilon)\}$. When
      making a move of the PDA, we have to choose one pair in its entirety; we
      cannot pick a state from one and a stack-replacement strings from another.

  \subsection*{A Graphical Notation for PDA's}
    Sometimes, a diagram, generalizing the transition diagram of a finite
    automaton, will make aspects of the behavior of a given PDA clearer. We
    shall therefore introduce and subsequently use a \emph{transition diagram}
    for PDA's in which:
      \begin{enumerate}
        \item[a)] The nodes correspond to the states of the PDA>
        \item[b)] An arrow labeled \emph{Start} indicates the start state, and
        doubly circled states are accepting, as for finite automata.
        \item[c)] The arcs correspond to transitions of the PDA in the following
        sense. An arc labeled $a$, $X/\alpha$ from state $q$ to state $p$ means
        that $\delta(q,a,X)$ contains the pair $(p,\alpha)$, perhaps among other
        pairs. That is, the arc label tells what input is used and also gives
        the old and new tops of the stacks.
      \end{enumerate}
    Conventionally, we use $Z_0$ as the start symbol for the stack.

  \subsection*{Instantaneous Descriptions of a PDA}
    Intuitively, the PDA goes from configuration to configuration, in response
    to input symbols (or sometimes $\epsilon$), but unlike the finite automaton,
    where the state is the only thing that we need to know about the automaton,
    the PDA's configuration involves both the state and the contents of the
    stack. Being arbitrarily large, the stack often more important part of the
    total configuration of the PDA at any time. \\
    \indent Thus, we shall represent the configuration of a PDA by a triple
    $(q,w,\gamma)$, where
      \begin{enumerate}
        \item $q$ is the state,
        \item $w$ is the remaining input, and
        \item $\gamma$ is the stack contents.
      \end{enumerate}
    Conventionally, we show the top of the stack at the left end of $\gamma$ and
    the bottom at the right end. Such a triple is called an \emph{instantaneous
    description}, or ID, of the pushdown automaton. \\
    \indent For finite automata, the $\hat{\delta}$ notation was sufficient to
    represent sequences of instantaneous descriptions through which a finite
    automaton moved since the ID for a finite automaton is just its state.
    However, for PDA's we need a notation that describes changes in the state,
    the input, and stack. Thus we adopt the ``turnstile'' notation for
    connection pairs of ID's that represent one or many moves of a PDA. \\
    \indent Let $P = (Q,\Sigma,\Gamma,\delta,q_0,Z_0,F)$ be a PDA. Define
    $\underset{P}{\vdash}$, or just $\vdash$ when $P$ is understood as follows.
    Suppose $\delta(q,a,X)$ contains $(p,\alpha)$. The for all strings $w$ in
    $\Sigma^*$ and $\beta$ in $\Gamma^*$:
      \[ (q,aw,X\beta)\vdash(p,w,\alpha\beta) \]
    This move reflects the idea that, by consuming $a$ (which may be $\epsilon$)
    from the input and replacing $X$ on top of the stack by $\alpha$, we can go
    from state $q$ to state $p$. \\
    \indent We also use the symbols $\underset{P}{\overset{*}{\vdash}}$, or $
    \overset{*}{\vdash}$ when the PDA $P$ is understood, to represent zero or
    more moves of the PDA. \\
    \indent There are three important principles about ID's and their
    transitions that we shall need in order to reason about PDA's:
      \begin{enumerate}
        \item If a sequence of ID's (\emph{computation}) is legal for a PDA $P$,
        then the computation formed by adding the same additional input string
        to the end of the input (second component) in each ID is also legal.
        \item If a computation is legal for a PDA $P$, then the computation
        formed by adding the same additional stack symbols below the stack in
        each ID is also legal.
        \item If a computation is legal for a PDA $P$, and some tail of the
        input is not consumed, then we can remove this tail from the input in
        each ID, and the resulting computation will still be legal.
      \end{enumerate}
    Intuitively, data that $P$ never looks at cannot affect its computation.
    Thus, we conclude:
      \begin{thm}
        If $P = (Q,\Sigma,\Gamma,\delta,q_0,Z_0,F)$ is a PDA, and $(q,x,\alpha)
        \overset{*}{\underset{P}{\vdash}} (p,y,\beta)$, then for any strings $w$
        in $\Sigma^*$ and $\gamma$ in $\Gamma^*$, it is also true that
          \[
            (q,xw,\alpha\gamma)\overset{*}{\underset{P}{\vdash}}(p,yw,\beta
            \gamma)
          \]
      \end{thm}
    Note that if $\gamma = \epsilon$, then we have a formal statement of
    principle (1) above, and if $w = \epsilon$, then we have the second
    principle.
      \begin{thm}
        If $P = (Q,\Sigma,\Gamma,\delta,q_0,Z_0,F)$ is a PDA, and
          \[
            (q,xw,\alpha)\overset{*}{\underset{P}{\vdash}}(p,yw,\beta)
          \]
        then it is also true that $(q,x,\alpha)\overset{*}{\underset{P}{\vdash}}
        (p,y,\beta)$.
      \end{thm}

\section*{The Languages of a PDA}
  We have assumed that a PDA accepts its input by consuming it and entering an
  accepting state. We call this approach ``acceptance by final state.'' There is
  a second approach to defining the language of a PDA that has important
  applications. We may also define for any PDA the language ``accepted by empty
  stack,'' that is, the set of strings that cause the PDA to empty its stack,
  starting from the initial ID. \\
  \indent These two methods are equivalent, in the sense that a language $L$ has
  a PDA that accepts it by final state if and only if $L$ has a PDA that accepts
  it by empty stack. However, for a given PDA $P$, the languages that $P$
  accepts by final state and by empty stack are usually different.

  \subsection*{Acceptance by Final State}
    Let $P = (Q,\Sigma,\Gamma,\delta,q_0,Z_0,F)$ be a PDA. Then $L(P)$, the
    \emph{language accepted by P by final state} is
      \[
        \{ w \, | \, (q_0,w,Z_0)\overset{*}{\underset{P}{\vdash}}(q,\epsilon,
        \alpha)\}
      \]
    for some state $q$ in $F$ and any stack string $\alpha$. That is, starting
    in the initial ID with $w$ waiting on the input, $P$ consumes $w$ from the
    input and enters an accepting state. The contents of the stack at that time
    is irrelevant.

  \subsection*{Acceptance by Empty Stack}
    For each PDA $P = (Q,\Sigma,\Gamma,\delta,q_0,Z_0,F)$, we also define
      \[
        N(P) = \{ w \, | \, (q_0,w,Z_0) \overset{*}{\vdash}(q,\epsilon,
        \epsilon)
      \]
    for any state $q$. That is, $N(P)$ is the set of inputs $w$ that $P$ can
    consume and at the same time empty its stack. \\
    \indent Since the set of accepting states is irrelevant, we shall sometimes
    leave off the last (seventh) component from the specification of a PDA $P$,
    if all we care about is the language that $P$ accepts by empty stack. Thus,
    we would write $P$ as a six-tuple $(Q,\Sigma,\Gamma,\delta,q_0,Z_0)$.

  \subsection*{From Empty Stack to Final State}
    \begin{thm}
      If $L = N(P_N)$ for some PDA $P_N = (Q,\Sigma,\Gamma,\delta_N,q_0,Z_0)$,
      then there is a PDA $P_F$ such that $L = L(P_F)$.
    \end{thm}

  \subsection*{From Final State to Empty Stack}
    \begin{thm}
      Let $L$ be $L(P_F)$ for some PDA
      $P_F = (Q,\Sigma,\Gamma,\delta_F,q_0,Z_0,F)$.
    \end{thm}

\section*{Equivalence of PDA's and CFG's}
  To prove that the languages defined by PDA's are exactly the context-free
  languages, we must prove the following three classes of languages:
    \begin{enumerate}
      \item The context-free languages, i.e., the languages defined by CFG's.
      \item The languages that are accepted by final state by some PDA.
      \item The languages that are accepted by empty stack by some PDA.
    \end{enumerate}
  are all the same class.

  \subsection*{From Grammars to Pushdown Automata}
    Given a CFG $G$, we construct a PDA that simulates the leftmost derivations
    of $G$. Any left-sentential form that is not a terminal string can be
    written as $xA\alpha$, where $A$ is the leftmost variable, $x$ is whatever
    terminals appear to its left, and $\alpha$ is the string of terminals and
    variables that appear to the right of $A$. We call $A\alpha$ the \emph{tail}
    of this left-sentential form. If a left-sentential form consists of
    terminals only, then its tail is $\epsilon$. \\
    \indent The idea behind the construction of a PDA from a grammar is to have
    the PDA simulate the sequence of left-sentential forms that the grammar uses
    to generate a given terminal string $w$. The tail of each sentential form
    $xA\alpha$ appears on the stack, with $A$ at the top. At that time, $x$ will
    be ``represented'' by our having consumed $x$ from the input, leaving
    whatever of $w$ follows its prefix $x$. That is, if $w = xy$, then $y$ will
    remain on the input. \\
    \indent Suppose that PDA is in an ID $(q,y,A\alpha)$, representing a
    left-sentential form $xA\alpha$. It guesses the production to use to expand
    $A$, say $A \rightarrow \beta$. The move of the PDA is to replace $A$ on the
    top of the stack by $\beta$, entering ID $(q,y,\beta\alpha)$. Note that
    there is only one state, $q$, for this PDA. \\
    \indent Now $(q,y,\beta\alpha)$ may not be a representation of the next
    left-sentential form, because $\beta$ may have a prefix of terminals. In
    fact, $\beta$ may have a prefix of terminals. Whatever terminals appear at
    the beginning of $\beta\alpha$ need to be removed, to expose the next
    variable at the top of the stack. These terminals are compared against the
    next input symbols, to make sure our guesses at the leftmost derivation of
    input string $w$ are correct; if not, this branch of the PDA dies. \\
    \indent If we succeed in this way to guess a leftmost derivation of $w$,
    then we shall eventually reach the left-sentential form $w$. At that point,
    all the symbols on the stack have either been expanded (if they are
    variables) or matched against the input (if they are terminals). The stack
    is empty, and we accept by empty stack. \\
    \indent Let $G = (V,T,Q,S)$ be a CFG. Construct the PDA $P$ that accepts
    $L(G)$ by empty stack as follows:
      \[ P = (\{q\},T,V\cup T,\delta,q,S) \]
    where transition function $\delta$ is defined by:
      \begin{enumerate}
        \item For each variable $A$,
          \[
            \delta(q,\epsilon,A) = \{(q,\beta) \, | \, A \rightarrow \beta \, \,
            \text{is a production of} \, \, G \}
          \]
        \item For each terminal $a$, $\delta(q,a,a) = \{(q,\epsilon)\}$.
      \end{enumerate}
      \begin{thm}
        If PDA $P$ is constructed from CFG $G$ by the construction above, then
        $N(P) = L(G)$.
      \end{thm}

  \subsection*{From PDA's to Grammars}
    Now, we complete the proofs of equivalence by showing that for every PDA
    $P$, we can find a CFG $G$ whose language is the same language that $P$
    accepts by empty stack. The idea behind the proof is to recognize that the
    fundamental event in the history of a PDA's processing of a given input is
    the net popping of one symbol off the stack, while consuming some input. A
    PDA may change state as it pops stack symbols, so we should also note the
    state it enters when it finally pops a level off its stack. \\
    \indent For example, some input $x_1$ is read while $Y_1$ is popped. We
    should emphasize that this ``pop'' is the net effect of (possibly) many
    moves. For example, the first move may change $Y_1$ to some other symbol
    $Z$. The next move may replace $Z$ by $UV$, later moves have the effect of
    popping $U$, and then other moves pop $V$. The net effect is that $Y_1$ has
    been replaced by nothing; i.e., it has been popped, and all the input
    symbols consumed so far constitute $x_1$. \\
    \indent We suppose that the PDA starts out in state $p_0$, with $Y_1$ at the
    top of the stack. After all the moves whose net effect is to pop $Y_1$, the
    PDA is in state $p_1$. It then proceeds to (net) pop $Y_2$, while reading
    input string $X_2$ and winding up, perhaps after many moves in state $p_2$
    with $Y_2$ off the stack. The computation proceeds until each of the symbols
    on the stack is removed. \\
    \indent Our construction of an equivalent grammar uses variables each of
    which represents an ``event'' consisting of:
      \begin{enumerate}
        \item The net popping of some symbol $X$ from the stack, and
        \item A change in state from some $p$ at the beginning to $q$ when $X$
        has finally been replaced by $\epsilon$ on the stack.
      \end{enumerate}
    We represent such a variable by the composite symbol $\lbrack pXq \rbrack$.
    Remember that this sequence of characters is our way of describing
    \emph{one} variables it is not five grammar symbols.
      \begin{thm}
        Let $P = (Q,\Sigma,\Gamma,\delta,q_0,Z_0)$ be a PDA. Then there is a
        context-free grammar $G$ such that $L(G) = N(P)$.
      \end{thm}

\section*{Deterministic Pushdown Automata}
  While PDA's are by definition allowed to be nondeterministic, the
  deterministic subcase is quite important. In particular, parsers generally
  behave like deterministic PDA's, so the class of languages that can be
  accepted by these automata is interesting for the insights it give us into
  what constructs are suitable for use in programming languages.

  \subsection*{Definition of a Deterministic PDA}
    Intuitively, a PDA is deterministic if there is never a choice of move in
    any situation. These choices are of two kinds. If $\delta(q,a,X)$ contains
    more than one pair, then surely the PDA is nondeterministic becuase we can
    choose mong these pairs when deciding on the next move. However, even if
    $\delta(q,a,X)$ is always a singleton, we could still have a choice between
    using a real input symbol or making a move on $\epsilon$. Thus, we define a
    PDA $P = (Q,\Sigma,\Gamma,q_0,Z_0,F)$ to be \emph{deterministic} (DPDA), if
    and only if the following conditions are met:
      \begin{enumerate}
        \item $\delta(q,a,X)$ has at most one member for any $q$ in $Q$, $a$ in
        $\Sigma$ or $a = \epsilon$, and $X$ in $\Gamma$.
        \item If $\delta(q,a,X)$ is nonempty, for some $a$ in $\Sigma$, then
        $\delta(q,\epsilon,X)$ must be empty.
      \end{enumerate}

  \subsection*{Regular Langauges and Deterministic PDA's}
    The DPDA's accept a class of languages that is between the regular languages
    and the CFL's.
      \begin{thm}
        If $L$ is a reuglar language, then $L = L(P)$ for some DPDA $P$.
      \end{thm}
    If we want the DPDA to accept by empy stack, then we find that our
    language-recognizing capability is rather limited. Say that language $L$ has
    the \emph{prefix property} if there are no two different strings $x$ and $y$
    in $L$ such that $x$ is a prefix of $y$.
      \begin{thm}
        A language $L$ is $N(P)$ for some DPDA $P$ if and only if $L$ has the
        prefix property and $L$ is $L(P^\prime)$ for some DPDA $P^\prime$.
      \end{thm}

  \subsection*{DPDA's and Context-Free Languages}
    The languages accepted by DPDA's by final state properly include the regular
    languages, but are properly included in the CFL's.

  \subsection*{DPDA's and Ambiguous Grammars}
    \begin{thm}
      If $L = N(P)$ for some DPDA $P$, then $L$ has an unambiguous context-free
      grammar.
    \end{thm}
    \begin{thm}
      If $L = L(P)$ for some DPDA $P$, then $L$ has an unambiguous CFG.
    \end{thm}

\section*{Summary of Chapter 6}
  \begin{description}
    \item[Pushdown Automata] A PDA is a nondeterministic finite automaton
    coupled with a stack that can be used to store a string of arbitrary lenght.
    The stack can be read and modified only at its top.
    \item[Moves of a Pushdown Automata] A PDA chooses its next move based on its
    current state, the next input symbol, and the symbol at the top of its
    stack. It may also choose to make a move independent of the input symbol and
    without consuming that symbol from the input. Being nondeterministic, the
    PDA may have some finite number of choices of move; each is a new state and
    a string of stack symbols with which to replace the symbol currently on top
    of the stack.
    \item[Acceptance by Pushdown Automata] There are two ways in whcih we may
    allow the PDA to signal acceptance. One is by entering an accepting state;
    the other by emptying its stack. These methods are equivalent, in the sense
    that any language accepted by one method is acceted (by some other PDA) by
    the other method.
    \item[Instantaneous Descriptions] We use an ID consisting of the state,
    remaining input, and stack contents to describe the ``current condition'' of
    a PDA. A transition function $\vdash$ between ID's represents sinle moves by
    a PDA.
    \item[Pushdown Automata and Grammars] The languages accepted by PDA's either
    by final state or by empty stack, are exactly the context-free languages.
    \item[Deterministic Pushdown Automata] The two modes of acceptance -- final
    state and empty stack -- are not the same for DPDA's. Rather, the languages
    accepted by empty stack are exactly those of the languages accepted by final
    state that have the prefix property: no string in the language is a prefix
    of another word in the language.
    \item[The Languages Accepted by DPDA's] All the regular languages are
    accepted (by final state) by DPDA's and there are nonregular languages
    accepted by DPDA's. The DPDA languages are context-free languages, and in
    fact are languages that have unambiguous CFG's. Thus teh DPDA languages lie
    strictly between the regular languages and the context-free languages.
  \end{description}
\end{document}
