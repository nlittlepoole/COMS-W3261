\documentclass[]{article}
\usepackage{amsmath}
\usepackage{amssymb}
\usepackage{amsthm}
\usepackage{indentfirst}
\usepackage{tikz}
\usetikzlibrary{arrows,automata}
\begin{document}

\title{COMS W3261 \\ Computer Science Theory \\ Chapter 1 Notes}
\author{Alexander Roth}
\date{2014-09-06}
\maketitle

\section*{Definitions}
  \begin{description}
    \item[Grammars] A useful model when designing software that processes data 
    with a recursive structure.
    \item[Regular Expressions] Denote the structure of data, especially text
    strings.
  
  \subsection*{Regular Expression Special Characters}
    \begin{description}
      \item[\texttt{\lbrack A-Z\rbrack}] Represents a range of characters from
      capital `A' to capital `Z'.
      \item[\texttt{\lbrack \, \rbrack}] Represents a blank character alone.
      \item[\texttt{*}] Represents ``any number of'' the preceding expression.
      \item[( \, )] Used to group components of the expression. 
    \end{description}
    
  \item[Decidability] What can a computer do at all? Problems that a computer
  can solve are called ``decidable''.
  \item[Intractability] What can a computer do efficiently? The problems that 
  can be solved by a computer using no more time than some slowly growing 
  function of the size of the input are called ``tractable.''
  
  \end{description}
 
\section*{Introduction to Formal Proofs}

  \subsection*{Reduction to Definitions}
    \begin{itemize}
      \item If you are not sure how to start a proof, convert all terms in the
      hypothesis to their definitions.
    \end{itemize}
    \begin{enumerate}
      \item A set $S$ is \emph{finite} if there exists an integer $n$ such that
      $S$ has exactly $n$ elements. We write $\|S\| = n$, where $\|S\|$ is used 
      to denote the number of elements in a set $S$.
      \item If the set $S$ is not finite, we say $S$ is \emph{infinite}.
      Intuitively, an infinite set is a set that contains more than any integer
      number of elements.
      \item If $S$ and $T$ are both subsets of some set $U$, then $T$ is the
      \emph{complement} of $S$ (with respect to $U$) if $S \cup T = U$ and
      $S \cap T = \emptyset$. That is each element of $U$ is in exactly one of
      $S$ and $T$; put another way, $T$ consists of exactly those elements of
      $U$ that are not in $S$.
    \end{enumerate}
    \begin{description}
      \item[Proof by Contradiction] Assume that the conclusion is false. Then
      use that assumption, together with parts of the hypothesis, to prove the
      opposite of one of the given statements of the hypothesis. Shown that it
      is impossible for all parts of the hypothesis to be true and for the
      conclusion to be false at the same time, there is only one possibility.
      The conclusion must be true whenever the hypothesis is true. Thus, the
      theorem is true.
    \end{description}
  
  \subsection*{Ways of Saying ``If-Then''}
    \begin{enumerate}
      \item $H$ implies $C$.
      \item $H$ only if $C$.
      \item $C$ if $H$.
      \item Whenever $H$ holds, $C$ follows.
    \end{enumerate}
      \subsubsection*{Examples}
        \begin{enumerate}
          \item $x \geq 4$ implies $2^x \geq x^2$.
          \item $x \geq 4$ only if $2^x \geq x^2$.
          \item $2^x \geq x^2$ if $x \geq 4$.
          \item Whenever $x \geq 4$, $2^x \geq x^2$ follows.
        \end{enumerate}
    The operator $\rightarrow$ can also take the place of ``if-then''.
    
  \subsection*{If-And-Only-If Statements}
    Forms of the statement ``$A$ if and only if $B$'' include: ``$A$ iff $B$'',
    ``$A$ is equivalent to $B$'' or ``$A$ exactly when $B$''. This statement is
    actually two if-then statements. We prove this by doing:
    \begin{enumerate}
      \item The \emph{if part}: ``if $B$ then $A$,'' and
      \item The \emph{only-if part}: ``if $A$ then $B$,'' which is often stated
      in the equivalent form ``$A$ only if $B$.''
    \end{enumerate}
    The operators $\leftrightarrow$ and $\equiv$ are used to denote 
    ``if-and-only-if'' statements.
  
  \subsection*{How Formal Do Proofs Have to Be?}
    There are certain things that are required in proofs, and omitting them
    surely makes the proof inadequate. For example, any deductive proof that
    uses statements which are not justified by given or previous statements,
    cannot be adequate.

\section*{Additional Forms of Proofs}
  \subsection*{Proving Equivalences About Sets}
    If $E$ and $F$ are two expressions representing sets, the statement
    $E = F$ means that the two sets represented are the same.
    
  \subsection*{The Contrapositive}
    The \emph{contrapositive} of the statement ``if $H$ then $C$'' is ``if not
    $C$ then not $H$''. A statement its contrapositive are either both true or
    both false, so we can prove either to prove the other.
    
  \subsection*{The Converse}
    The \emph{converse} of an if-then statement is the ``other direction''; 
    that is, the converse of ``if $H$ then $C$'' is ``if $C$ then $H$.'' 
    Unlike the contrapositive, which is logically equivalent to the original, 
    the converse is \emph{not} equivalent to the original statement.
  
  \subsection*{Counterexamples}
    Statements that have no parameters, or that apply to only a finite number
    of values of its parameter(s) are called \emph{observations}. It is often
    easier to prove that a statement is not a theorem than to prove it 
    \emph{is} a theorem.

\end{document}