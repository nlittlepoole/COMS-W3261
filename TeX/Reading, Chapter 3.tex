\documentclass[]{article}
\usepackage{amsmath}
\usepackage{amssymb}
\usepackage{amsthm}
\usepackage{indentfirst}
\usepackage{tikz}
\usetikzlibrary{arrows,automata}
\begin{document}

\title{COMS W3261 \\ Computer Science Theory \\ Chapter 3 Notes}
\author{Alexander Roth}
\date{2014-09-09}
\maketitle

\section*{Regular Expressions}
  Regular expressions can define exactly the same languages that the various
  forms of automata describe: the regular languages. However, regular
  expressions offer something that automata do not: a declarative way to express
  the strings we want to accept. Thus regular expressions serve as the input
  language for many systems that process strings.

  \subsection*{The Operators of Regular Expressions}
    Regular expressions denote languages. There are three operations on
    languages that the operators of regular expressions represent. These
    operators are:
    \begin{enumerate}
      \item The \emph{union} of two languages $L$ and $M$, denoted by
      $L \cup M$, is the set of strings that are in either $L$ or $M$ or both.
      \item The \emph{concatenation} of languages $L$ and $M$ is the set of
      strings that can be formed by taking any string in $L$ and concatenating
      it with any string in $M$. The concatenation operator is frequently called
      ``dot''.
      \item The \emph{closure} (or \emph{star}, or \emph{Kleene closure}) of a
      language $L$ is denoted $L^*$ and represents the set of those strings that
      can be formed by taking any number of strings from $L$, possibly with
      repetitions and concatenating all of them.
    \end{enumerate}

\section*{Applications of Regular Expressions}
  We shall consider two important classes of regular-expression-based
  applications: lexical analyzers and text search.

  \subsection*{Regular Expressions in UNIX}
    Before seeing the applications, we shall introduce the UNIX notation for
    extended regular expressions. This notation gives us a number of additional
    capabilities. In fact, the UNIX extensions include certain features,
    especially the ability to name and refer to previous strings that have
    matched a pattern, that actually allow nonregular languages to be
    recognized. \\
    \indent The first enhancement to the regular-expression notation concerns
    the fact that most real applications deal with the ASCII character set. UNIX
    regular expressions allow us to write \emph{character classes} to represent
    large sets of characters as succinctly as possible. The rules for character
    classes are:
      \begin{itemize}
        \item The symbol \texttt{.} (dot) stands for ``any character''.
        \item The sequence $[a_1a_2\cdot{}a_k]$ stands for the regular
        expression

          \[ a_1 + a_2 + \cdots + a_k \]

        The notation saves about half the characters, since we don't have to
        write the $+-$ signs. For example, we could express the four characters
        used in C comparison operators by \texttt{[<>=!]}.
        \item Between the square braces we can put a range of the form
        \emph{x-y} to mean all the characters from $x$ to $y$ in the ASCII
        sequence. Since the digits have codes in order, as do the upper-case
        letters and the lower-case letters, we can express many of the classes
        of characters that we really care about with just a few keystrokes. If
        we want to include a minus sign among a list of characters, we can place
        it first or last, so it is not confused with its use to form a character
        range. Square brackets, or other characters that have special meanings
        in UNIX regular expressions can be represented as characters by
        preceding them with a backslash (\textbackslash).
        \item There are special notations for server of the most common classes
        of characters. For instance:
        \begin{enumerate}
          \item[a)] \texttt{[:digit:]} is the set of ten digits, the same as
          \texttt{[0-9]}.
          \item[b)] \texttt{[:alpha:]} stands for any alphabetic character, as
          does \texttt{[A-Za-z]}.
          \item[c)] \texttt{[:alnum:]} stands for the digits and letters
          (alphabetic and numeric characters), as does \texttt{[A-Za-z0-9]}.
        \end{enumerate}
      \end{itemize}

    In addition, there are several other operators that are used in UNIX
    regular expressions. None of these operators extended what languages can
    be expressed, but they sometimes make it easier to express what we want.
      \begin{itemize}
        \item The operator | is used in place of $+$ to denote union.
        \item The operator \texttt{?} means ``zero or one of.'' Thus, $R$? in
        UNIX is the same as $\epsilon + R$ in our notation.
        \item The operator $+$ means ``one or more of.'' Thus, $R+$ in UNIX is
        shorthand for $RR^*$ in our notation.
        \item The operator $\{ n \}$ means ``$n$ copies of.'' Thus, $R\{5\}$ in
        UNIX is shorthand for $RRRRR$.
      \end{itemize}

\end{document}
