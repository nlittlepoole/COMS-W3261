\documentclass[]{article}
\usepackage[all]{xy}
\usepackage{amsmath}
\usepackage{amssymb}
\usepackage{amsthm}
\usepackage{enumitem}
\usepackage{indentfirst}
\usepackage{listings}
\usepackage{multirow}
\usepackage{tikz}
\usepackage{tikz-qtree}
\usepackage{tipa}
\usetikzlibrary{arrows,automata}
\begin{document}

\newtheorem{thm}{Theorem}
\title{COMS W3261 \\ Computer Science Theory \\ Homework                                                              \#2                           }
\author{Alexander Roth}
\date{2014 -- 09 -- 29}
\maketitle

\section*{Problems}
  \begin{enumerate}
    \item Consider the grammar $G$:
      \[ S \rightarrow a S b S \, | \, b S a S \, | \epsilon \]
      \begin{enumerate}
        \item Describe in English the language generated by $G$. Prove that $G$
        generates precisely this language.
        \item[\emph{Solution:}]
          The language generated by $G$ returns all even strings of $a$'s and
          $b$'s such that the number of $a$'s is equal to the number of $b$'s.
        \item[\textbf{PROOF:}] We shall proove that a string $w$ which is an
        even length string of the same number of $a$'s and $b$'s (order-
        independent) is in $L(G)$ if and only if it is an even length string
        with an equal number of $a$'s  and $b$'s. \\
        (If) Suppose $w$ is an even length string of equal numbers of $a$'s and
        $b$'s. We show by induction on $|w|$ that $w$ is in $L(G)$.
        \item[\textbf{BASIS:}] We use lengths 0 and 2 as the basis. If $|w|$ =
        0 or $|w|$ = 2, then $w$ is $\epsilon$, $ab$, or $ba$. We do not use a
        length of 1 because $w$ is an even length string. Since there
        are productions $S \rightarrow \epsilon$, $S \rightarrow aSbS$, and $S
        \rightarrow bSaS$, we conclude that $S \overset{*}{\Rightarrow} w$ in
        any of these basis cases.
        \item[\textbf{INDUCTION:}] Suppose $|w| \geq 2$. Since $w$ is an even
        length string of equal number of $a$'s and $b$'s, we know that for
        every $a$ in the string, there is an matching $b$. That is, $w = aSbS$
        or $w = bSaS$. Moreover, we know that $x$ must be an even length string
        of equal numbers of $a$'s and $b$'s or an empty string ($\epsilon$).
        Note that we need the fact that $|w| \geq 2$ to infer that there is a
        distinct $a$ paired with a distinct $b$ within $w$. \\ \\
        If $w = aSbS$, then we invoke the inductive hypothesis to claim that
        $G\overset{*}{\Rightarrow} S$. Then there is is a derivation of $w$
        from $S$, namely $S \Rightarrow aSbS \overset{*}{\Rightarrow} axbx = w$
        If $w = bSaS$, the argument is the same, but we use the production
        $S \rightarrow bSaS$ at the first step. In either case, we conclude
        that $w$ is in $L(G)$ and complete the proof. \\ \\
        (Only-if) Now, we assume that $w$ is in $L(G)$; that is,
        $S \overset{*}{\Rightarrow} w$. We must conclude that $w$ is an even
        length string of equal number of $a$'s and $b$'s. The proof is an
        induction on the number of steps in a derivation of $w$ from $S$.
        \item[\textbf{BASIS:}] If the derivation is one step, then it must use
        one of the three productions that do not have $S$ in the body. That is,
        the derivation of $S \Rightarrow \epsilon$, $S \Rightarrow ab$, or
        $S \Rightarrow ba$ Since $\epsilon$, $ab$, and $ba$, are all even
        length strings of equal $a$'s and $b$'s, the basis is proven.
        \item[\textbf{INDUCTION:}] Now, suppose that the derivation takes $n+1$
        steps, where $n \geq 1$, and the statement is true for all derivations
        of $n$ steps. That is, if $S \overset{*}{\Rightarrow} x$ in $n$ steps,
        then $x$ is an even length string of equal numbers of $a$'s and $b$'s.
        \\\\
        Consider an $(n + 1)$-step derivation of $w$, which must be of the form
          \[ S \Rightarrow aSbS \overset{*}{\Rightarrow} axbx = w \]
        or $S \Rightarrow bSaS \overset{*}{\Rightarrow} bxax = w$, since
        $n + 1$ steps is at least two steps, and the productions
        $S \rightarrow aSbS$ and $S \rightarrow bSaS$ are the only productions
        whose use allows additional steps of a derivation. Note that in either
        case, $S \overset{*}{\Rightarrow} x$ in $n$ steps. \\\\
        By the inductive hypothesis, we know that $x$ is an even length string
        of equal number of $a$'s and $b$'s.  But if so, then $axbx$ and $bxax$
        are also even length strings with equal numbers of $a$'s and $b$'s.
        Thus, we conclude that $w$ is an even length string of equal $a$'s and
        $b$'s, which completest the proof $\qed$
        \item Is the language generated by $G$ regular? Prove your answer.
        \item[\emph{Solution:}] Let us show that $L(G)$ consisting of all
        strings with an equal number of $a$'s and $b$'s is not a regular
        language. Suppose $n$ is the constant that must exist if $L(G)$ is
        regular, according to the pumping lemma. Thus, $w = a^nb^n$, that is $n
        $ $a$'s followed by $n$ $b$'s. A string which is definitely in $L(G)$.
        \\\\
        Let us break up the string $w$ into three parts: $x$, $y$, and $z$. We
        know that $y \neq \epsilon$ and that $|xy| \leq n$, and that
        $xy$ comes at the front of $w$. Since $|xy| \leq n$ and $xy$ comes at
        the front of $w$, we know that $x$ and $y$ consist of only $a$'s. The
        pumping lemma tells us that $xz$ is in $L(G)$, if $L(G)$ is regular and
        if $k = 0$. However, $xz$ has $n$ $b$'s because all the $b$'s of $w$
        are in $z$. But $xz$ also has fewer than $n$ $a$'s, because we lost the
        $a$'s of $y$. Since $y \neq \epsilon$, we know that there can be no
        more than $n - 1$ $a$'s among $x$ and $z$. Thus, after assuming $L(G)$
        is a regular language, we have proved that $xz$ cannot be in $L(G)$.
        Thus, this is a proof by contradiction that $L(G)$ cannot be a regular
        language.
        \item Demonstrate whether $G$ is ambiguous or unambiguous.
        \item[\emph{Solution:}] Consider the sentential form $abab$ It has 2
        derivations:
          \begin{enumerate}
            \item[1.] $S \Rightarrow aSbS \Rightarrow abS \Rightarrow abaSbS
            \Rightarrow ababS \Rightarrow abab$
            \item[2.] $S \Rightarrow aSbS \Rightarrow abSaSbS
            \overset{*}{\Rightarrow} abab$
          \end{enumerate}
        These derivations can be represented as two distinct parse trees:
          \Tree [.$S$ [.$a$ ] [.$S$ [.$\epsilon$ ] ] [.$b$ ] [.$S$ [.$a$ ]
          [.$S$ [.$\epsilon$ ] ] [.$b$ ] [.$S$ [.$\epsilon$ ] ] ] ]
          \Tree [.$S$ [.$a$ ] [.$S$ [.$b$ ] [.$S$ [.$\epsilon$ ] ] [.$a$ ]
          [.$S$ [.$\epsilon$ ] ] ] [.$b$ ] [.$S$ [.$\epsilon$ ] ] ] \\
        Since there are two distinct parse trees for the same yield, we regard
        this grammar as ambiguous.
      \end{enumerate}
    \item Put the grammar $G$ in question (1) into Chomsky Normal Form. Use the
    Cocke-Younger-Kasami algorithm to parse the sentence \texttt{abab}
    according to your CNF grammar. Display the CYK table constructed by the CYK
    algorithm and show how all parse trees for this sentence can be
    reconstructed from the CYK table.
    \item[\emph{Solution:}] Grammar $G$ is:
      \[ S \rightarrow aSbS \, | \, bSaS \, | \, \epsilon \]
    First, we eliminate all $\epsilon$-productions. We need to find the
    nullable symbols. $S$ is directly nullable because it has a production with
    $\epsilon$ as the body. \\\\
    Now, let us construct the productions of grammar $G_1$. First, consider
    $S \rightarrow aSbS$. The second and fourth positions hold nullable
    symbols, so there are four choices of present/absent. Our four choices
    yield:
      \[ S \rightarrow aSbS \, | \, aSb \, | \, abS \, | \, ab \]
    Similarly, the second production for $G_1$ yields:
      \[ S \rightarrow bSaS \, | \, bSa \, | \, baS \, | \, ba \]
    Thus, we have the following productions for $G_1$:
      \[
        S \rightarrow aSbS \, | aSb \, | \, abS \, | \, ab \, | bSaS \, | \,
        bSa \, | \, baS \, | \, ba
      \]
    There are no unit pairs in this grammar, so we don't have to worry about
    those. Now, we check for useless states. All symbols are generating so
    there are no useless states. \\\\
    Now we can begin to put $G_1$ into Chomsky Normal Form. First, we give
    variables to the terminals $a$ and $b$; that is,
      \[ A \rightarrow a \quad B \rightarrow b \]
    If we look at grammar $G_1$, we now have
      \[
        S \rightarrow ASBS \, | \, ASB \, | \, ABS \, | \, AB \, | \, BSAS \, |
        \, BSA \, | \, BAS \, | \, BA
      \]
    Now, we break apart bodies of length 3 or more into a cascade of productions
    like so
      \[ C \rightarrow AS \quad D \rightarrow BS \]
    Grammar $G_1$ is now
      \[
        S \rightarrow CD \, | \, CB \, | \, AD \, | \, AB \, | \, DC \, | \, DA
        \, | \, BC \, | BA
      \]
    This gives us the Chomsky Normal Form for grammar $G$. Let us use the
    Cocke-Younger-Kasami algorithm to parse the sentence \texttt{abab} according
    to the CNF grammar.

    {
      \centering
      \begin{tabular}{|cccc}
        $\{S\}$    &            &            &            \\
        $\{C\}$    & $\{D\}$    &            &            \\
        $\{S\}$    & $\{S\}$    & $\{S\}$    &            \\
        $\{A\}$    & $\{B\}$    & $\{A\}$    & $\{B\}$    \\ \hline
        \texttt{a} & \texttt{b} & \texttt{a} & \texttt{b}
      \end{tabular} \\
    }
    The CYK table for the CNF of $G$. We can reconstruct the parse trees as
      \Tree [.$S$ [.$C$ [.$A$ [.$a$ ] ] [.$S$ [.$B$ [.$b$ ] ]
      [.$A$ [.$a$ ] ] ] ] [.$B$ [.$b$ ] ] ]
      \qquad
      \Tree [.$S$ [.$A$ [.$a$ ] ] [.$D$ [.$B$ [.$b$ ] ] [.$S$ [.$A$ [.$a$ ] ]
      [.$B$ [.$b$ ] ] ] ] ] \\
    By following the path given to us by the CYK algorithm.
    \item From the grammar $G$ in question (1), construct a pushdown automaton
    that accepts $L(G)$ by empty stack. Show all sequences of moves that your
    PDA can make to accept the input string \texttt{abab}.
    \item[\emph{Solution}:] From grammar $G$, we construct the pushdown automata
    $P$. Recall that the grammar $G$ is:
      \[ S \rightarrow aSbS \, | \, bSaS \, | \, \epsilon \]
    The set of input symbols for the PDA is $\{a, b, \epsilon \}$. These three
    symbols and $S$ form the stack alphabet.  The transition function for the
    PDA is:
      \begin{enumerate}
        \item $\delta(q,\epsilon,S) = \{(q,aSbS), (q,bSaS), (q, \epsilon)\}$
        \item $\delta(q,a,a) = \{(q,\epsilon)\}$;
        $\delta(q,b,b) = \{(q,\epsilon)\}$;
        $\delta(q,\epsilon,\epsilon) = \{(q,\epsilon)\}$.
      \end{enumerate}
    Thus, for PDA $P$, we have
      \[ P = (\{q\},\{a,b,\epsilon\},\{a,b,\epsilon,S\},\delta,q,S) \]
    where $\delta$ is the transition function described above. We omit $F$, the
    set of \emph{accepting states}, since $P$ accepts by empty stack. \\\\
    We now show all sequences of moves that $P$ can make to accept the input
    string \texttt{abab}. Arrows represent the $\vdash$ relation. Note: I am
    showing only the two accepting paths. if I were to show all paths, the diagram
    would probably spread over 5 pages. \\\\
      \begin{displaymath}
         \xymatrix{
           (q,\texttt{abab},S)    \ar[d]                                   \\
           (q,\texttt{abab},aSbS) \ar[d]                                   \\
           (q,\texttt{bab} ,SbS)  \ar[d]                           \ar[dr] \\
           (q,\texttt{bab} ,bS)   \ar[d] & (q,\texttt{bab},bSaSbS) \ar[d]  \\
           (q,\texttt{ab}  ,S)    \ar[d] & (q,\texttt{ab},SaSbS)   \ar[dl] \\
           (q,\texttt{ab}  ,aSbS) \ar[d]                                   \\
           (q,\texttt{b}   ,SbS)  \ar[d]                                   \\
           (q,\texttt{b}   ,bS)   \ar[d]                                   \\
           (q,\epsilon, S)        \ar[d]                                   \\
           (q,\epsilon, \epsilon)
        }
      \end{displaymath}
    \item From your PDA in question (3), construct an equivalent context-free
    grammar. Show a parse tree for the input string \texttt{abab} according to
    your grammar.
    \item[\emph{Solution}:] Let us convert the PDA
    $P = (\{q\},\{a,b,\epsilon\}, \{a,b,\epsilon,S\},\delta,q,S)$ into a CFG
    $G_2$. There are only two variables in the grammar $G_2$:
      \begin{enumerate}
        \item $S$, the start symbol.
        \item $\lbrack qSq \rbrack$, the only triple that can be assembled from
        the states and stack symbols of $P$.
      \end{enumerate}
    The productions for grammar $G_2$ are as follows:
      \begin{enumerate}
        \item[1.] The only production for $S$ is
        $S \rightarrow \lbrack qSq \rbrack$.
        \item[2.] From the fact that $\delta(q,\epsilon,S)$ contains $(q,aSbS)$,
        we have $\lbrack qSq \rbrack \rightarrow a\lbrack qSq\rbrack{b}
        \lbrack{qSq}\rbrack$. Similarly, since $\delta(q,\epsilon,S)$ contains $
        (q,bSaS)$ we have $\lbrack{qSq}\rbrack\rightarrow b\lbrack{qSq}\rbrack{a}
        \lbrack{qSq}\rbrack$. Finally, we have $\lbrack{qSq}\rbrack\rightarrow
        \epsilon$
      We may, for convenience, replace the triple, $\lbrack{qSq}\rbrack$, by some
      less complex symbol, $A$. Thus, our productions become
        \begin{align*}
          S &\rightarrow A \\
          A &\rightarrow bAaA \, | \, aAbA \, | \, \epsilon
        \end{align*}
      We can express these productions as
        \[
          G_2 = (\{S\}, \{a,b,\epsilon\},\{S\rightarrow aSbS \, | \, bSaS \,|
          \,\epsilon\}, S)
        \]
      The parse tree of string $\texttt{abab}$ according to grammar $G_2$ is \\
        \Tree [.$S$ [.$a$ ] [.$S$ [.$b$ ] [.$S$ [.$\epsilon$ ] ] [.$a$ ]
        [.$S$ [.$\epsilon$ ] ] ] [.$b$ ] [.$S$
        [.$\epsilon$ ] ] ]
        \qquad
        \Tree [.$S$ [.$a$ ] [.$S$ [.$\epsilon$ ] ] [.$b$ ] [.$S$ [.$a$ ]
        [.$S$  [.$\epsilon$ ] ]
        [.$b$ ] [.$S$ [.$\epsilon$ ] ] ] ]
      \end{enumerate}
    \item Consider the language $L$ consisting of all strings of $a$'s and
    $b$'s that are even-length palindromes with the same number of $a$'s as
    $b$'s. If $L$ is context free, construct a CFG for $L$ and prove that your
    grammar generates precisely this language. If $L$ is not context free, use
    the pumping lemma for context-free languages to prove that $L$ is not
    context free.
    \item[\emph{Solution}:] We have a language $L$ which consists of all strings
    of $a$'s and $b$'s which are even-length palindromes with the same number of
    $a$'s and $b$'s. Let us choose $z = a^nb^nb^na^n$, where $n$ is the number of
    $a$'s and $b$'s. This string is certainly in the language $L$.
    Next, $z$ is broken up into 5 parts: $uvwxy$, with the constraints
    being $|vwx| \leq n$, and $vx \neq \epsilon$. Thus, we have three cases
      \begin{enumerate}
        \item[1.] $vwx$ is the string of all $a$'s before the $b$'s in the
        palindrome.
        \item[2.] $vwx$ is the string of all $b$'s after the set of all $a$'s in
        the palindrome.
        \item[3.] $vwx$ is the string of $a$'s and $b$'s in $z$.
      \end{enumerate}
      \begin{description}
        \item[\textbf{CASE 1:}] Since $|vwx| \leq n$, $vwx$ is in the string of $a
        $'s either located at the beginning or end of the palindrome. Since $vx
        \neq \epsilon$, if we look at the case where $i = 0$, we have $v^0$ and
        $x^0$; thus, we are losing all the $a$'s within $v$ and $x$. Therefore, $z
        $ does not have an equal number of $a$'s and $b$'s. This statement causes
        a contradiction since $z$ is in language $L$.
        \item[\textbf{CASE 2:}] $vwx$ is the strings of $b$'s in the palindrome;
        that is, $vwx$ is in either the first $b^n$ or the second $b^n$, since
        $|vwx| \leq n$. Since $vx \neq \epsilon$, the case of $i = 0$ implies that
        there are less than $2n$ $b$'s while there are still $2n$ $a$'s in the
        string $z$. Therefore, there is not the same number of $a$'s and $b$'s.
        Again, this is a contradiction since we stated $z$ is in language $L$,
        when it clearly cannot be.
        \item[\textbf{CASE 3:}] $vwx$ is the string of $a$'s and $b$'s in $z$ such
        that $|vwx| \leq n$. Since $vx \neq \epsilon$, if we look at the case
        where $i = 0$, we see that $z$ can no longer be a palindrome since there
        will be fewer than $n$ $a$'s and fewer than $n$ $b$'s on one side of
        string $z$. Thus, $z$ cannot be in $L$ and this provides us with a
        contradiction.
      \end{description}
    Whichever case holds, we conclude that $L$ has a string we know not to be in
    $L$. This contradiction allows us to conclude that our assumption was wrong;
    $L$ is \emph{not} a CFL. $\qed$
  \end{enumerate}
\end{document}
