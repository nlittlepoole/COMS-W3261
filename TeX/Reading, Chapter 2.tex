\documentclass[]{article}
\usepackage{amsmath}
\usepackage{amssymb}
\usepackage{amsthm}
\usepackage{indentfirst}
\usepackage{tikz}
\usetikzlibrary{arrows,automata}
\begin{document}

\title{COMS W3261 \\ Computer Science Theory \\ Chapter 2 Notes}
\author{Alexander Roth}
\date{2014-09-06}
\maketitle

\section*{Definitions}
  \begin{description}
    \item[Deterministic] The automaton cannot be in more than one state at any 
    time.
    \item[Nondeterministic] The automaton may be in several states at once.
  \end{description}

\section*{An Informal Picture of Finite Automata}
  We will be using the example of a real-world problem whose solution uses 
  finite automata in an important role. We investigate protocols that support 
  ``electronic money'' -- files that a customer can use to pay for goods on 
  the internet, and that the seller can receive insurance that the ``money'' 
  is real.
  
  \subsection*{The Ground Rules}
    There are three participants: the customer, the store, and the bank. We 
    assume for simplicity that there is only one ``money'' file in existence.
    Interactions among the three participants is thus limited to five 
    elements:
    \begin{enumerate}
      \item The customer may decide to \emph{pay}. That is, the customer sends
      the money to the store.
      \item The customer may decide to \emph{cancel}. The money is sent to the
      bank with a message that the value of the money is to be added to the
      customer's bank account.
      \item The store may \emph{ship} goods to the customer.
      \item The store may \emph{redeem} the money. That is, the money is sent
      to the bank with a request that its value be given to the store.
      \item The bank may \emph{transfer} the money by creating a new, suitably
      encrypted money file and sending it to the store.
    \end{enumerate}
    
  \subsection*{The Protocol}
    The customer cannot be relied upon to act responsibly. In particular, the 
    customer may try to copy the money file, use it to pay several times, or 
    both pay and cancel the money, thus getting the goods ``for free''. The 
    bank must behave responsibly, or it cannot be a bank. The store should be 
    careful as well. In particular, it should not ship goods until it is sure 
    it has been given valid money for the goods. \\
    \indent These types of protocols can be represented by finite automata. 
    Each state represents a situation that one of the participants could be 
    in. Transitions between states occur when one of the five events described 
    above occur.


\end{document}