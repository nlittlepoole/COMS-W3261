\documentclass[]{article}
\usepackage[all]{xy}
\usepackage{amsmath}
\usepackage{amssymb}
\usepackage{amsthm}
\usepackage{enumitem}
\usepackage{indentfirst}
\usepackage{hhline}
\usepackage{listings}
\usepackage{multirow}
\usepackage{tikz}
\usepackage{tikz-qtree}
\usepackage{tipa}
\usetikzlibrary{arrows,automata}
\begin{document}

\newtheorem{thm}{Theorem}
\title{COMS W3261 \\ Computer Science Theory \\ Homework \#3}
\author{Alexander Roth}
\date{2014 -- 10 -- 27}
\maketitle
\section*{Problems}
\begin{enumerate}
\item Informally describe a Turing machine that accepts all strings of the form
$\{\,a^nb^nc^n\,|\,n\geq1\,\}$. Show the sequence of ID's that your TM goes
through starting with the input $aabbcc$.
\item[\emph{Solution:}] Let us construct the TM that will accept the language
$\{\,a^nb^nc^n\,|\,n\geq1\,\}$. Initially, it is given a finite sequence of
$a$'s, $b$'s, and $c$'s on its tape, preceded and followed by an infinity of
blanks. Alternatively, the TM will change an $a$ to an $X$, a $b$ to a $Y$, and
a $c$ to a $Z$, until all $a$'s, $b$'s and $c$'s have been matched.

Starting at the left end of the tape, it will change an $a$ to an $X$, it then
moves to the right passing over any $a$'s and $Y$'s it encounters. When it comes
upon a $b$, it will change that into a $Y$ and continue to the right, ignoring
any subsequent $b$'s and $Z$'s. When it comes across a $c$, it transforms the
$c$ into a $Z$ and continue to the right. When it reads a $B$, it will move to
left, over any $Z$'s, $c$'s, $Y$'s, $b$'s, and $a$'s it encounters. When it
reaches an $X$, it looks for an $a$ to the immediate right. If that is found, it
repeats the process till it accepts. If, instead, the Turing machine reads a Y,
it will continue to read down the string to the right. If it reads past the
string and reads a blank, it accepts the string, as there all $a$'s, $b$'s and
$c$'s have been matched.

The turing machine looks like this:
\[ M = (\{q_0, q_1, q_2, q_3, q_4, q_5\}, \{a, b, c\}, \{a, b, c, X, Y, Z, B\},
\delta, q_0, B, \{q_5\} \]
where the transition function is given by the following table:\\

\begin{tabular}{c|ccccccc}
      & \multicolumn{7}{c}{Symbol} \\
State &$a$        &$b$        &$c$        &$X$&$Y$        &$Z$        &$B$
\\ \hhline{========}
$q_0$ &$(q_1,X,R)$&$-$        &$-$
&$-$&$(q_0,Y,R)$&$(q_0,Z,R)$&$(q_5,B,L)$ \\
$q_1$ &$(q_1,a,R)$&$(q_2,Y,R)$&$-$        &$-$&$(q_1,Y,R)$&$-$        &$-$ \\
$q_2$ &$-$        &$(q_2,b,R)$&$(q_3,Z,R)$&$-$&$-$        &$(q_2,Z,R)$&$-$ \\
$q_3$ &$-$        &$-$        &$(q_3,c,R)$&$-$&$-$        &$-$
&$(q_4,B,L)$\\
$q_4$
&$(q_4,a,L)$&$(q_4,b,L)$&$(q_4,c,L)$&$(q_0,X,R)$&$(q_4,Y,L)$&$(q_4,Z,L)$&$-$
\\
$q_5$ &$-$     &$-$     &$-$     &$-$     &$-$     &$-$     &$-$     \\
\end{tabular} \\

Thus, for the string $aabbcc$, we have the following sequence:
\begin{align*}
&q_0aabbcc\vdash{Xq_1abbcc}\vdash{Xaq_1bbcc}\vdash{XaYq_2bcc}\vdash{XaYbq_2cc}\v
dash \\
&XaYbZq_3c\vdash{XaYbZcq_3B}\vdash{XaYbZq_4cB}\vdash{XaYbq_4ZcB}\vdash{XaYq_4bZc
B}\vdash \\ &Xaq_4YbZcB\vdash{Xq_4aYbZcB}\vdash{q_4XaYbZcB}\vdash{Xq_0aYbZcB}\vd
ash{XXq_1YbZcB}\vdash\\&XXYq_1bZcB\vdash{XXYYq_2ZcB}\vdash{XXYYZq_2cB}\vdash{XXY
YZZq_3B}\vdash{XXYYZq_4ZB}\vdash\\&XXYYq_4ZZB\vdash{XXYq_4YZZB}\vdash{XXq_4YYZZB
}\vdash{Xq_4XYYZZB}\vdash{XXq_0YYZZB}\vdash\\&XXYq_0YZZB\vdash{XXYYq_0ZZB}\vdash
{XXYYZq_0ZB}\vdash{XXYYZZq_0B}\vdash{XXYYZq_5ZB}
\end{align*}
Since we have entered state $q_5$, we accept the string $aabbcc$.

\item Consider the following Turing machine
\[ M = (\{A, B, C, D\}, \{a\}, \{a, X, 0, 1, \#\}, \delta, A, \#, \{D\}).\]
Here, we are using \# for the blank symbol.
The transition function $\delta$ is given by the following table:

\begin{tabular}{c|c|c|c|c|c }
State & $a$   & $X$    & $0$   & $1$   & $\#$   \\
$A$   & $BXL$ & $AXR$  & $A0R$ & $A1R$ & $C\#L$ \\
$B$   & $BaL$ & $BXL$  & $A1R$ & $B0L$ & $A1R$  \\
$C$   & $-$   & $C\#L$ & $D0R$ & $D1R$ & $-$    \\
$D$   & $-$   & $-$    & $-$   & $-$   & $-$
\end{tabular}
\begin{enumerate}
\item Show the sequence of ID's that $M$ goes through starting with the input
$aaaa$.
\item[\emph{Solution:}]
\begin{align*}
&Aaaaa\vdash{B\#Xaaa}\vdash{1AXaaa}\vdash{1XAaaa}\vdash{1BXXaa}\vdash\\&B1XXaa\v
dash{B\#0XXaa}\vdash{1A0XXaa}\vdash{10AXXaa}\vdash{10XAXaa}\vdash\\&10XXAaa\vdas
h{10XBXXa}\vdash{10BXXXa}\vdash{1B0XXXa}\vdash{11AXXXa}\vdash\\&11XAXXa\vdash{11
XXAXa}\vdash{11XXXAa}\vdash{11XXBXX}\vdash{11XBXXX}\vdash\\&11BXXXX\vdash{1B1XXX
X}\vdash{B10XXXX}\vdash{B\#00XXXX}\vdash{1A00XXXX}\vdash\\&10A0XXXX\vdash{100AXX
XX}\vdash{100XAXXX}\vdash{100XXAXX}\vdash{100XXXAX}\vdash\\&100XXXXA\#\vdash{100
XXXCX\#}\vdash{100XXCX\#\#}\vdash{100XCX\#\#\#}\vdash{100CX\#\#\#\#}\vdash\\&10C
0\#\#\#\#\#\vdash{100D\#\#\#\#\#}
\end{align*}
$M$ halts on the final state $D$ when given the string $aaaa$ as input; thus,
$aaaa$ is accepted.

\item Starting with an input consisting of $n$ $a$'s, $n > 0$, what string will
this Turing machine have on its tape after it has halted?
\item[\emph{Solution:}] The string that will be on the tape of the Turing
machine will consist of a binary representation of $n$. For example, a string
consisting of 3 $a$'s (aaa) would yield a string on the tape as $11\#\#\#\#$.
Since we now represent the blank symbol as $\#$, the number of $\#$ is
insignificant.

\item Briefly explain how the Turing machine does this computation and
characterize the role of each state.
\item[\emph{Solution:}] This Turing machine repeatedly finds its rightmost $a$
and replaces that $a$ with an $X$. It then moves to the left, searching for a
$\#$, 0, or 1. If it finds a $\#$, it will write a 1 over the $\#$, start moving
to the right, passing over any X's, O's, or 1's until it comes upon another $a$.
This process continues until it reads off the rightmost index of the input
string, at which point it reads a $\#$ (provided this is a continuous string of
$a$'s). It will then convert all the placeholder $X$'s to blank symbols and halt
when it reaches a 0 or a 1.\\

\textbf{The States}
\begin{description}
\item[$A$:] This state begins the cycle, and breaks it when appropriate. The
Turing machine will scan to the right for an $a$, at which point it will
transition to state $B$ after writing an $X$ over that $a$. It will pass over
any $0$'s or $1$'s found on the tape. If at any point it reads a blank symbol
while in this state, it will transition to state $C$, and pass over the blank.
\item[$B$:] In this state, the Turing machine searches to the left for any $0$'s
or $\#$. If it reaches a $0$, it converts it to a $1$ and enters state $A$. If
it reads a $\#$, it transforms that into a $1$; both allowing the process to
repeat. If it moves over a $0$, it will remain in state $B$, but it will change
the 0 to a 1.
\item[$C$:] At this stage, the Turing machine is reading the blank symbol
adjacent to the rightmost input symbol. Thus, it will read over the string,
transforming any $X$ it comes across into blanks. Once it reads a $0$ or a $1$,
it transitions to state $D$.
\item[$D$:] The sole purpose of this state is to allow the Turing machine to
halt when it has finished its task.
\end{description}

\item Using big-O notation, how many moves will this Turing machine make on an
input consisting of $n$ $a$'s before halting? Briefly justify your answer.
\item[\emph{Solution}:] This Turing machine operates in $O(n^2)$ time. Suppose
we have a string of $n$ $a$'s. It will locate the $n$th $a$ after traversing
down the string and adjusting the counter $n - 1$ times. Thus, it will take
roughly $n$ times to go through a string of length $n$. The time taken to adjust
the binary string is approximately $n\log(n)$. Finally, there are $n$ moves to
transform all $X$'s at the end of the string to blank symbols. However, since we
are using Big-$O$ notation, only the $n^2$ matters.s
\end{enumerate}

\item Let $L = \{\,p\,|\,p$ is a polynomial over a single variable $x$ with an
integral root $\}$. (Example: $2x^3-9x^2+16$ is a polynomial over $x$ with an
integral root $4$ and would therefore be in $L$.) Describe at an informal level
a Turing machine $M$ such that $L(M) = L$ showing that $L$ is recursively
enumerable.
\item[\emph{Solution}:] We shall construct a Turing machine $M$ that takes as
input the polynomial $p$. It will move along the tape and in the process set $x$
to successive values (e.g., 0, 1, -1, 2, \ldots). $M$ will evaluate $p$ over
these different values of $x$. If at any point $p$ evaluates to 0, $M$ will halt
and accept. However, if $p$ has a root that is not integral, $M$ will never
halt.

\item Post's Correspondence Problems.
\begin{enumerate}
\item Is PCP with a single-symbol alphabet decidable? Briefly justify your
answer.
\item[\emph{Solution}:] The PCP with a single-symbol alphabet is decidable.
Since we are using only a single-symbol alphabet, all that matters is the length
of each string in the pair. Thus, we can define $A_i$ and $B_i$ to be two
strings that form the pairs for the PCP problem where $i$ represents the number
of pairs in the string. We can assume that there must be at least two pairs in
the problem; otherwise, the solution would be decidable if $A = B$ and for no
other solutions. Therefore, we have a four cases we must account for:
\begin{description}
\item[Case 1] $|A_i| = |B_i|$ for at least one pair in the problem. Thus, this
pair provides a solution to the problem.
\item[Case 2] $|A_i| > |B_i|$ for all pairs in the problem. Thus, the length of
strings formed for $A$ will always be longer than those strings that are formed
for $B$. From this scenario, we can see that there is no solution. Without loss
of generality, we can switch $A_i$ with $B_i$ and find similar results (i.e.,
there is a string with a longer length in $B$ than $A$ for all pairs; thus,
there is no solution).
\item[Case 3] $|A_i| > |B_i|$ for at least one pair in the problem, and $|B_j| >
|A_j|$ for at least one other pair. We can find some solution to this case by
\end{description}

\item Is PCP with a two-symbol alphabet decidable? Briefly justify your answer.
\item[\emph{Solution}:] The PCP with a two-symbol alphabet is undecidable.
Suppose we have the original PCP with an $n$-sized alphabet and a modified PCP
with a binary alphabet. We shall reduce the first PCP to the instance of PCP
over the binary alphabet. We apply the homomorphism that maps a symbol
$\alpha_i$ as $1^i0$, where $i$ is the length of 1's formed from the index of
the symbol. That is, for any instance of the PCP over the $n$-sized alphabet, we
have a mapping to the binary alphabet. From this, we have reduced the PCP over
an $n$-sized alphabet can be reduced to a binary PCP. Since the $n$-sized PCP is
undecidable, we know that the binary PCP is undecidable.
\end{enumerate}

\item What class of languages can a Turing machine recognize if it
\begin{enumerate}
\item Has only two working states, and one accepting state from which it never
makes any transitions?
\item[\emph{Solution}:] This language is recursively enumerable. We can
construct this Turing machine to simulate any other Turing machine by mapping
the states of the larger Turing machine to the states of this machine.
Furthermore, we will have this machine use the same alphabet. Thus, the new
Turing machine will simulate all the actions of the larger Turing machine and
halt when the original machine would have halted.

\item Never overprints a different symbol on the input tape? That is, if in the
transition function for the Turing machine whenever $(q, Y, D)$ is in
$\delta(p,X)$, then $Y = X$.
\item[\emph{Solution}:] This language is regular. Since it cannot overwrite any
of the symbols, we know that it cannot have any memory of past actions; that is,
it behaves like a DFA. This machine will only act on the next input without
regard to what it previous saw, until it reaches some accepting state, when it
will then halt.

\item Has only $\{0, 1, B\}$ as tape symbols?
\item[\emph{Solution}:] This language is recursively enumerable. We can
represent any Turing machine with this alphabet by using some homomorphism that
maps it  s alphabet to a set of binary strings generated from $\{0, 1, B\}$.
Thus, we can encode any other language in binary.

\item Never moves its input head left?
\item[\emph{Solution}:] This language is regular. Under this restriction, the
Turing machine behaves like a DFA. That is, it maintains no memory of what it
previously saw, and acts on the next input it reads from the tape, until it
reaches some halting state. Thus, it is regular.
\end{enumerate}

Give a brief one or two-sentence justification for each of your answers.
\end{enumerate}

\end{document}
