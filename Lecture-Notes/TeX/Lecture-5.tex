\documentclass[]{article}
\usepackage{amsmath}
\usepackage{amssymb}
\usepackage{amsthm}
\usepackage{listings}
\usepackage{multirow}
\usepackage{tikz}
\usepackage{tikz-qtree}
\usepackage{tipa}
\usetikzlibrary{arrows,automata}
\begin{document}

\title{COMS W3261 \\ Computer Science Theory \\ Lecture 5\\ Properties of
Regular Expressions}
\author{Alexander Roth}
\date{2014 -- 09 -- 17}
\maketitle

\section*{Overview}
  \begin{itemize}
    \item Algebraic laws can be used to simplify regular expressions.
    \item The pumping lemma for regular languages can be used to prove that some
    languages are not regular.
    \item The set of regular languages is closed under many common operations
    such as union, intersection, complement and reversal.
  \end{itemize}

\section{Algebraic Laws for Regular Expressions}
  \begin{itemize}
    \item Algebraic laws can be used to simplify regular expressions
    \item Here are some of the most important algebraic identities for regular
    expressions
      \begin{itemize}
        \item Union is commutative: $L + M = M + L$
        \item Union is associative: $(L + M) + N = L + (M + N)$
        \item Concatenation is associative: $(LM)N = L(MN)$
        \item $\varnothing$ is the identity for union: $\varnothing + L = L +
        \varnothing = L$
        \item $\epsilon$ is the identity for concatenation: $\epsilon{L} = L
        \epsilon = L$
        \item $\varnothing$ is the annihilator for concatenation: $
        \varnothing{L} = L\varnothing = \varnothing$
        \item Concatenation left distributes over union: $L(M + N) = LM + LN$
        \item Concatenation right distributes over union: $(M + N)L = ML + NL$
        \item Union is idempotent: $L + L = L$
        \item $L^{**} = L^*$
        \item $\varnothing^{*} = \epsilon$
        \item $\epsilon^* = \epsilon$
      \end{itemize}
  \end{itemize}
  \subsection{Class Notes}
    Simplifying regular expressions:
    \begin{align*}
      & \epsilon + a + (\epsilon + a)(\epsilon + a)^*(\epsilon + a) \\
      & \epsilon + a + (\epsilon + a)a^*(\epsilon + a)              \\
      & \epsilon + a + a^*                                          \\
      & a^*
    \end{align*}

\section{The Pumping Lemma for Regular Languages}
  \begin{itemize}
    \item The pumping lemma for regular languages states that for every
    nonfinite regular language $L$, there exists a constant $n$ that depends on
    $L$ such that for all $w$ in $L$ with $|w| \geq n$, there exists a
    decomposition of $w$ into $xyz$ such that
      \begin{enumerate}
        \item $y \neq \epsilon$
        \item $|xy| \leq n$, and
        \item for all $k \geq 0$, the string $xy^kz$ is in $L$.
      \end{enumerate}
    \item Proof: See HMU, p. 129.
    \item One important use of the pumping lemma is to prove some languages are
    not regular.
    \item Example: The language $L$ consisting of all strings of a's and b's of
    the form $a^ib^i$, $i \geq 0$, is not regular.
      \begin{itemize}
        \item The proof will be by contradiction. Assume $L$ is regular. Then by
        the pumping lemma there is a constant $n$ associated with $L$ such that
        for all $w$ in $L$ with $|w| \geq n$, $w$ can be written as $xyz$ such
        that
          \begin{enumerate}
            \item $y \neq \epsilon$
            \item $|xy| \leq n$, and
            \item for all $k \geq 0$, the string $xy^kz$ is in $L$.
          \end{enumerate}
        \item Since $|xy| \leq n$, $xy = a^,$ for some $0 < m \leq n$.
        \item Setting $k = 0$, condition (3) of the pumping lemma says $xz$ must
        also be in $L$.
        \item But $xz$ is of the form $a^pb^n$, where $p < n$.
        \item This contradicts the conclusion that $xz$ must be in $L$.
      \end{itemize}
  \end{itemize}

\section{Closure Properties of Regular Languages}
  \begin{itemize}
    \item A closure property for a family of languages is a theorem that says if
    we apply a certain operation to the languages in the family, then the
    resulting language will also be in the family. For example. if we take the
    union of two regular languages $L$ and $M$, then the language $L \cup M$ is
    also regular. We therefore say the regular languages are closed under the
    operation of union.
    \item We can show that the regular languages are closed under the following
    operations:
      \begin{itemize}
        \item union, intersection, complement, difference
        \item concatenation, Kleene closure
        \item reversal
        \item homomorphism, inverse homomorphism
      \end{itemize}
    \item These closure properties can be used to show that some languages are
    regular.
    \item These closure properties combined with the pumping lemma can be used
    to show some languages are not regular.
  \end{itemize}

\section{Practice Problems}
  \begin{enumerate}
    \item Show that the language consisting of all strings of balanced
    parentheses is not regular.
    \item Prove that the language consisting of all strings of $a$'s and $b$'s
    that read the same forwards as backwards is not regular.
    \item Prove that the language $L = \{ w \, | \, w = a^ib^i$ where $i$ is not
    equal to $j \}$ is not regular.
  \end{enumerate}

\section{Reading Assignment}
  \begin{itemize}
    \item HMU: Sects. 3.4, 4.1, 4.2
  \end{itemize}
\end{document}
