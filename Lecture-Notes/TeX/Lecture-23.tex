\documentclass[]{article}
\usepackage[all]{xy}
\usepackage{amsmath}
\usepackage{amssymb}
\usepackage{amsthm}
\usepackage{enumitem}
\usepackage{indentfirst}
\usepackage{listings}
\usepackage{multirow}
\usepackage{tikz}
\usepackage{tikz-qtree}
\usepackage{tipa}
\begin{document}

\newtheorem{thm}{Theorem}
\title{Computer Science Theory \\ COMS W3261 \\ Lecture 23}
\author{Alexander Roth}
\date{2014 -- 12 -- 1}
\maketitle

\section*{Outline}
\begin{enumerate}
\item Church numerals
\item Arithmetic
\item Logic
\item Other programming language constructs
\item The influence of the lambda calculus on functional languages
\end{enumerate}

\section{Church Numerals}
\begin{itemize}
\item Church numbers are a way of represent the intergers in lambda calculus.
\item Church numbers are defined as functions taking two parameters
\begin{align*}
&0\,\,\textrm{is defined as}\,\,\lambda f.\lambda x.x \\
&1\,\,\textrm{is defined as}\,\,\lambda f.\lambda x.f\,x\\
&2\,\,\textrm{is defined as}\,\,\lambda f.\lambda x.f(f\,x)\\
&3\,\,\textrm{is defined as}\,\,\lambda f.\lambda x.f(f(f\,x))\\
&n\,\,\textrm{is defined as}\,\,\lambda f.\lambda x.f^n\,x
\end{align*}
\item $n$ has the property that for any lambda expressions $g$ and $y$, $ngy
\overset{*}{\rightarrow} g^ny$. That is to say, $ngy$ causes $g$ to be applied
to $y$ $n$ times.
\end{itemize}

\section{Arithmetic}
\begin{itemize}
\item In lambda calculus, arithmetic functions can be represented by
corresponding operations on Church numerals.
\item We can define a successor function $succ$ of three arguments that adds one
to its first argument:
\[ \lambda n.\lambda f.\lambda x.f (n\,f,x) \]
\begin{itemize}
\item Example: Let us evaluate $succ\,\,2 = $
\begin{align*}
&(\lambda n.\lambda f.\lambda x.f(n\,f\,x))(\lambda f^\prime.\lambda
x^\prime.f^\prime(f^\prime\,x^\prime)) \\
&\rightarrow \lambda f.\lambda x.f((\lambda f^\prime.\lambda
x^\prime.f^\prime(f^\prime\,x^\prime))f\,x) \\
&\rightarrow \lambda f.\lambda x.f(\lambda x^\prime.f(f\,x^\prime)x) \\
&\rightarrow \lambda f.\lambda x.f(f(f\,x)) \\
&= 3
\end{align*}
\end{itemize}
\item We can define a function $add$ as follows:
\[ \lambda m.\lambda n.\lambda f.\lambda x.m\,f(n\,f\,x) \]
\begin{itemize}
\item Example: Let us evaluate $add\,\,0\,1=$
\begin{align*}
&(\lambda m.\lambda n.\lambda f.\lambda x.m\,f(n\,f\,x)0\,1 \\
&\rightarrow \lambda n.\lambda f.\lambda x.0\,f\,(n\,f\,x)\,1\\
&\rightarrow \lambda f.\lambda x.0\,f(1\,f\,x) \\
&= \lambda f.\lambda x.(\lambda f^\prime.\lambda x^\prime.x^\prime)f(1\,f\,x)\\
&\rightarrow \lambda f.\lambda x.\lambda x^\prime.x^\prime(1\,f\,x) \\
&\rightarrow \lambda f.\lambda x.(1\,f\,x) \\
&= \lambda f.\lambda x.((\lambda f^\prime.\lambda
x^\prime.f^\prime\,x^\prime)f\,x) \\
&\rightarrow \lambda f.\lambda x.(\lambda x^\prime.f\,x^\prime)x\\
&\rightarrow \lambda f.\lambda x.f\,x \\
&= 1
\end{align*}
\end{itemize}
\item We can define a function $mult$ as follows:
\[ \lambda m.\lambda n.\lambda f.m(n\,f) \]
\begin{itemize}
\item Example: Let us evaluate $mul\,\,2\,3=$
\begin{align*}
&(\lambda m.\lambda n.\lambda f.m(n\,f))2\,3 \\
&\rightarrow \lambda n.\lambda f.2(n\,f)3 \\
&\rightarrow \lambda f.2(3\,f) \\
&\overset{*}{\rightarrow}\lambda f.\lambda x.f(f(f(f(f(f\,x))))) \\
&= 6
\end{align*}
\end{itemize}
\end{itemize}

\section{Logic}
\begin{itemize}
\item The boolean value true can be represented by a function of two arguments
that always selects its first argument: $\lambda x.\lambda y.x$
\item The boolean value false can be represented by a function of two arguments
that always selects its second argument: $\lambda x.\lambda y.y$
\item An if-then-else statement can be represented by a function of three
arguments $\lambda c.\lambda i.\lambda e.c\,i\,e$ that uses its condition $c$ to
select either the if-part $i$ or the else-part $e$.
\begin{itemize}
\item Example: Let us evaluate if true then 1 else 0:
\begin{align*}
&(\lambda c.\lambda i.\lambda e.c\,i\,e) \textrm{true}\,1\,0 \\
&\rightarrow(\lambda i.\lambda e.\textrm{true}\,i\,e)1\,0\\
&\rightarrow(\lambda e.\textrm{true}\,1\,e)0\\
&\rightarrow\textrm{true}\,1\,0\\
&\rightarrow(\lambda x.\lambda y.x)1\,0\\
&\rightarrow(\lambda y.1)0\\
&\rightarrow1
\end{align*}
\end{itemize}
\item The boolean operators and, or, and not can be implemented as follows:
\begin{align*}
\textrm{and} &= \lambda p.\lambda q.p\,q\,p\\
\textrm{or}  &= \lambda p.\lambda q.p\,p\,q\\
\textrm{not} &= \lambda p.\lambda a.\lambda b.p\,b\,a\\
\end{align*}
\begin{itemize}
\item Example: Let us evalue $\textrm{not true}$:
\begin{align*}
&(\lambda p.\lambda a.\lambda b.p\,b\,a)\textrm{true}
&\rightarrow \lambda a.\lambda b.\textrm{true}\,b\,a\\
&= \lambda a.\lambda b.(\lambda x.\lambda y.x)\,b\,a\\
&\rightarrow \lambda a.\lambda b.(\lambda y.b)a\\
&\rightarrow \lambda a.\lambda b.b\\
&= \textrm{false (under renaming)}
\end{align*}
\end{itemize}
\end{itemize}

\section{Other Programming Language Constructs}
\begin{itemize}
\item We can readily implement other programming language constructs in lambda
calculus. As an example, here are lambda calculus expressions for various list
operations such as cons (constructing a list), head (selecting the first item
from a list), and tail (selecting the remainder of a list after the first item):
\begin{align*}
\textrm{cons} &= \lambda h.\lambda t.\lambda f.f\,h\,t\\
\textrm{head} &= \lambda l.l (\lambda h.\lambda t.h)\\
\textrm{tail} &= \lambda l.l (\lambda h.\lambda t.t)\\
\end{align*}
\end{itemize}

\section*{Class Notes}
\subsection*{Church Numerals}
\begin{align*}
0 &= \lambda f.\lambda x.x \\
1 &= \lambda f.\lambda x.f\,x \\
2 &= \lambda f.\lambda x.f(f\,x)
\end{align*}

\subsubsection*{Example}
\[(\lambda f.(\lambda x.(f\,x)))E\,F \]
All functions are unary.
\begin{align*}
&\underset{\beta}{\rightarrow} (\lambda x.(E\,x))F \\
&\underset{\beta}{\rightarrow} (E\,F)
\end{align*}

\subsection*{Logic}
and true false
\begin{align*}
&((\lambda p.\lambda q.p\,q\,p)\textrm{true})\textrm{false}) \\
&\underset{\beta}{\rightarrow}(\lambda
q.\textrm{true}\,q\,\textrm{true})\textrm{false} \\
&\underset{\beta}{\rightarrow}\textrm{true false true}\\
&=((\lambda x.\lambda y.x)\textrm{false true}) \\
&=(\lambda y.\textrm{false})\textrm{true} \\
&=\textrm{false}
\end{align*}
Two lambda expressions are equivalent if you can rename one and substitute in
the other with bound variables.

\end{document}
