\documentclass[]{article}
\usepackage{amsmath}
\usepackage{amssymb}
\usepackage{amsthm}
\usepackage{listings}
\usepackage{multirow}
\usepackage{tikz}
\usepackage{tikz-qtree}
\usepackage{tipa}
\usetikzlibrary{arrows,automata}
\begin{document}
\newcommand*{\xml}[1]{\texttt{<#1>}}
\theoremstyle{definition}
\newtheorem{thm}{Theorem}
\title{COMS W3261 \\ Computer Science Theory \\ Lecture 20 \\ NP and Co-NP}
\author{Alexander Roth}
\date{2014 -- 11 -- 17}
\maketitle

\section*{Outline}
\begin{enumerate}
\item Review: SAT is NP-Complete
\item CSAT is NP-Complete
\item 3SAT is NP-Complete
\item Co-NP
\item SAT and SMT Solvers
\end{enumerate}

\section{Review: SAT is NP-Complete}
\begin{itemize}
\item SAT is the language $\{\,E\,|\,E$ is a satisfiable boolean expression
$\,\}$.
\item We have shown that SAT is NP-complete by (1) showing SAT is in NP and (2)
showing that every language in NP can be reduced to SAT in polynomial time.
\end{itemize}

\section{CSAT is NP-Complete}
\begin{itemize}
\item A boolean expression is in conjunctive normal form (CNF) if it is the
logical AND (conjunction) of clauses where a clause is the logical OR of one or
more literals. (Colloquially, the AND of ORs.)
\item We define CSAT as the language $\{\,E\,|\,E$ is a satisfiable CNF boolean
expression $\,\}$.
\item We can prove that CSAT is NP-Complete.
\begin{itemize}
\item First, we need to show that CSAT is in NP. But this is easy because a CNF
boolean expression is a special case of a boolean expression and SAT is in NP.
\item Second, we need to show we can reduce an instance of $E$ of SAT into an
instance $F$ of CSAT in polynomial time such that $E$ is satisfiable iff $F$ is
satisfiable. However, this part of the proof is a bit more complicated because
we need to do the reduction in polynomial time.
\begin{itemize}
\item Two boolean expressions are equivalent if they have the same value on
every truth assignment.
\item However, just converting a boolean expression $E$ to an equivalent CNF
boolean expression $E^\prime$ may make $E^\prime$ exponentially longer and thus
take exponential time (in the length of $E$).
\item To avoid this exponential blow-up, we can transform $E$ into another not
necessarily equivalent boolean expression $F$ but with the property that $E$ is
satisfiable iff $F$ is satisfiable. Moreover, we can do this transformation in
time polynomial in the length of $E$. Section 10.3 of HMU has the details.
\end{itemize}
\end{itemize}
\end{itemize}

\section{3SAT is NP-Complete}
\begin{itemize}
\item A boolean expression is in $k$-CNF if it is the logical AND of clauses
each one of which is the logical OR of exactly $k$ distinct literals. for
example, $(w\vee\neg x\,\vee y)\wedge(x\vee\neg y \vee z)$ is in 3-CNF.
\item We define $k$SAT as the language $\{\,E\,|\,E$ is a satisfiable $k$-CNF
boolean expression $\,\}$.
\item We can show that 1SAT and 2SAT are in P but $k$SAT is NP-Complete for $k
\geq 3$.
\end{itemize}

\section{Co-NP}
\begin{itemize}
\item P is closed under complement but is not known whether NP is closed under
complement.
\item CO-NP is the set of languages whose complements are in NP.
\item A language is co-NP-Complete if it is in Co-NP and if every language in
Co-NP is polynomial-time reducible to it.
\begin{itemize}
\item The language consisting of boolean expressions that are tautologies is Co-
NP-Complete.
\end{itemize}
\item It is widely believed that none of the NP-complete languages have their
complements in NP.
\item Thus, it is believed no NP-Complete language is in Co-NP.
\begin{itemize}
\item We suspect NP $\neq$ Co-NP because we can prove NP = Co-NP iff there is
some NP-complete problem whose complements is in NP.
\end{itemize}
\end{itemize}

\section{SAT and SMT Solvers}
\begin{itemize}
\item Even though SAT is NP-Complete (a worst-case result), software tools
called SAT solvers have been developed using various algorithms and heuristics
that seem to work well for instances of large practical SAT problems arising in
areas such as electronic design automation, model checking, software
verification, and artificial intelligence.
\item Satisfiability Modulo Theories (SMT) are generalizations of SAT where an
SMT instance is a formula in first-order logic in which some function and
predicate symbols have additional interpretations. A simple example would be an
instance of SAT in which some of the binary variables are replaced by linear
inequalities. Many SMT solvers have been built. In a typical use of an SMT
solver, we would translate various kinds of assertions attached to a program
into SMT formulas to see if the assertions are always true. SMT solvers allow a
richer set of questions to be posed than is possible with SAT solvers.
\end{itemize}

\section*{Class Notes}
Proving an NP-Complete problem is NP-Complete:
Suppose we have a problem $L^\prime$. We must:
\begin{enumerate}
\item show that $L^\prime$ is in $\mathcal{NP}$.
\item Reduce $L^\prime$ to an NP-complete problem $L$.
\end{enumerate}
CNF is the AND of a bunch of clauses. (The AND of OR's).

Consider $E = x_1y_1 + x_2y_2 + x_3y_3$ in disjunctive normal form. Map into
conjunctive normal form. Transform into an equivalent boolean expression $F$
\begin{align*}
F = &(x_1+x_2+x_3)(y_1+x_2+x_3)(x_1+y_2+x_3)\wedge \\
    &(y_1+y_2+x_3)(x_1+x_2+y_3)(x_1+x_2+y_3)\wedge \\
    &(x_1+y_2+y_3)(y_1+y_2+y_3)
\end{align*}
OR of $n$ AND's + each AND has 2 vars. AND of $2^n$ clauses of $n$ vars.

Step 1.
\begin{align*}
E &= \neg((\neg(x+y))(\overline{x}+y)) \\
&\Rightarrow \neg(\neg(x+y))+\neg(\overline{x}+y) \\
&\Rightarrow x+y + \neg(\overline{x}+y) \\
&\Rightarrow x + y + x\overline{y}
\end{align*}

Step 2. Transform the resulting expression $E$ of length $n$ into an expression
$F$ such that
\begin{enumerate}
\item $F$ is in CNF and has at most $n$ clauses.
\item $F$ is constructible from $E$ in $O(|E|)$ time.
\item A truth assignment $T$ for $E$ makes $E$ satisfiable iff there exists an
extension $S$ of $T$ that makes $F$ true.
\end{enumerate}

\end{document}
